\section{Test statistics correlation may not converge to population correlation}\label{chap2}
	\vspace*{5cm}
	\begin{centering}
		{\normalsize Bin Zhuo, Jiang Duo and Yanming Di}\\[.04\textheight]
	\end{centering}	
	\vspace*{5cm}

\newpage
	\begin{abstract}
	In this paper, we investigate the relationship between test statistics correlation and 
	population correlation. We are motivated by some research works where a large number of test 
	statistics are pooled together for the purpose of testing multiple genes or gene sets. Among 
	them, some methods are intended to adjust for correlation between test statistics, whose 
	correlation are estimated by sample correlation---a consistent estimator of the population 
	correlation. We show that test statistics correlation may not equal population correlation 
	unless the test statistic takes some specific form. Under bivariate normal assumption, we 
	present the exact formula for test statistics correlation when 
	the statistics are derived from equal-variance two sample $t$-test. Based on the formula,  we 
	prove that test statistics correlation is no more greater in absolute value than population 
	correlation.
	
	\end{abstract}

	\subsection{Introduction}
	%	\textbf{What's the consequence if the correlation between statistics cannot be represented 
	%sample correlation?}
	%	\begin{enumerate}
	%		\item Methods relating FDR control in terms of type I error seems to be OK?? Because 
	%under the null, test statistics correlation are (almost) the same as sample correlation.
	%		\item What about power in terms of FDR control?
	%		\item Competitive gene set test would definitely be affected, in terms of both type I 
	%error and power.  
	%		\item it seems, according to Efron's 2007 paper, that conditional FDR will also be 
	%affected. 
	%	\end{enumerate}
	%	
	%	\textbf{What problem do we address in this paper?}\\
	%
	%	
	%	\textbf{Introduction}\\
	
	Inter-gene correlations are commonly observed in sequencing data generated from gene expression 
	experiments \citep{efron2012large1, gatti2010heading, 
	huang2013gene,qiu2005effects,storey2003positive}.
	The key task of expression analysis is to detect differentially expressed (DE) genes whose 
	expression levels are associated with experimental or treatment variables under study. 
	In such a task, 
	a summary statistic is calculated for each gene to quantify the magnitude of DE. The test 
	statistics are often of familiar form. For example, 
	they may come from two-sample comparison or experimental design based regression models. 
	However, since the expression levels are correlated, the test statistics calculated from the 
	expression levels are also correlated \citep{barry2008statistical, efron2007correlation, 
		wu2012camera}. This paper concerns the relationship between test 
	statistics correlations and the corresponding expression level correlations.
	
%	\textbf{Why would people care about correlation between genes?}\\
	The stochastic dependence of test statistics has brought methodological issues to statistical 
	models that are intended to test individual genes or gene sets and assume independence among 
	test statistics. The interest in examining individual genes is to find DE genes among tens of 
	thousands of candidates. Multiple hypothesis testing 
	procedures, such as \textit{false discovery rate} (FDR) \citep{benjamini1995controlling} and 
	\textit{$q$-value} \citep{storey2003positive}, 
	are needed to control type I error rate. In many cases, such techniques work only 
	when test statistics are independent \citep{benjamini1995controlling} or 
	have positive regression dependency \citep{benjamini2001control}. The goal of evaluating gene 
	sets is to find molecular pathways or gene 
	networks that are related to the experimental condition or factors of interest. Testing a gene 
	set is usually done by pooling the test 
	statistics of its member genes, and may or may not involve genes not in the test set 
	\citep{goeman2007analyzing}. In all situations, the 
	correlation between test statistics is a nuisance aspect, which, if not addressed 
	appropriately, will undermine the applicability of the 
	corresponding approaches \citep{gatti2010heading, wu2012camera}. For
	example, \citet{efron2007correlation} showed in a simulation study that for a nominal FDR of 
	$0.1$, the actual FDR can easily vary by a factor of 10 when correlation between test 
	statistics 
	exists. 
	
%	\textbf{What are existing ways of dealing with inter-gene correlations?}\\
	A number of attempts have been made to deal with issues of inter-gene correlation when testing 
	either individual genes or gene sets. One 
	approach is to derive certain summary statistic from correlation among test statistics and 
	then use it in the hypothesis testing procedure. For testing individual genes, 
	\citet{efron2007correlation} estimates some dispersion variate to summarize correlation among 
	test statistics, and then calculates the \textit{false discovery 
		proportion} (FDP) conditioning on this dispersion variate.  For 
	testing gene sets, \citet{wu2012camera} estimate a \textit{variance inflation factor} (VIF) 
	associated with inter-gene correlation and 
	incorporate it into their parametric/rank-based gene set test procedures. The same VIF is also 
	used by \citet{yaari2013quantitative} to account 
	for correlation in their distribution-based gene set test. Another approach is to permute the 
	labels of biological samples, aiming to generate the null distribution of test statistic for 
	each gene.
	This type of permutation preserves underlying correlation structure between genes, and thus 
	protect the test against such correlations. The 
	\textit{gene set enrichment analysis} (GSEA) procedure \citep{subramanian2005gene} falls into 
	this category.
	However, sample permutation method has an extra assumption, which states that the test 
	statistics always follow the distribution they have under complete null  that no gene is DE 
	\cite{efron2012large1}. In other words, this 
	assumption expects that the distribution of test statistics under the null is not affected by 
	the presence of non-null cases. For this reason, we will not discuss sample permutation based 
	methods in this paper.  
	
%	\textbf{Key question: Are  expression level correlations the same as test statistics 
%	correlation?}\\
	Summarizing test statistics correlation requires that the correlations between test statistics 
	are known or at least can be estimated from the data. Without 
	replicating the experiment, however, there's no way to obtain the correlation between any pair 
	of test statistics because only a single  statistic is available for each gene. In the case of 
	one-sided test (e.g., two sample $t$-test), one possible choice is to use sample 
	correlations (after gene treatment effects nullified) to represent correlations among test 
	statistics \citep{barry2008statistical, efron2007correlation, wu2012camera, 
		yaari2013quantitative}. 
	%The sample correlation is a consistent estimator of underlying population correlation 
	%\cite{fisher1915frequency}. 
	\citet{efron2007correlation} estimates the distribution of $z$-value (transformed from 
	corresponding two sample $t$-test statistic) 
	correlation by sample correlation. \citet{barry2008statistical} show by Monte Carlo simulation 
	of gene expression data that a nearly linear 
	relationship holds between test statistic correlation and sample correlation for several types 
	of test statistics they examine.  This Monte Carlo simulation results are cited by 
	\citet{wu2012camera} as a justification for estimating their VIF---a summary of correlation 
	between test statistics---from sample correlation.
	In all of the 
	works, it is shown by simulation only the 
	equivalence (in terms of either distribution or numerical summarization) of sample correlation  
	and test statistics	correlation. 
	To the best of our knowledge, such equivalence has not yet been justified or disproved 
	theoretically.
	
	
%	\textbf{What did we find}\\
	We investigate the effect of testing procedures on inter-gene correlation. First, we present a 
	formula for calculating correlation between test statistics when they take specific form and 
	meet some assumption of independence. Then we apply this formula to a special case where 
	two-group comparison experiment is considered. We show that 1) the test statistics correlation 
	$\rho_T$ is equal to the population correlation $\rho$ when the test statistics are a linear 
	combination of the expression levels, and that 2) $\rho_T$ is no more larger than $\rho$ in 
	absolute value when the test statistics are derived from two sample $t$ test. We conduct 
	simulations to illustrate our findings.
	
	
%	\textbf{Relevant but different work}\\
	A relevant research was done by \citet{qiu2005effects}, in which they studied the effect of 
	different
	normalization procedures on the inter-gene correlation structure for microarray data. They 
	randomly
	assigned 330 arrays into 15 pairs, each containing 22 arrays within each array 12558 genes. 
	Then 15
	$t$-statistics were calculated for each gene to mimic 15 two-sample comparisons under null
	hypothesis of no DE. They compared the histogram of $t$-statistics correlation for different
	normalization algorithms, and concluded that the normalization procedures are unable to 
	completely
	remove the correlation between the test statistics. % In this work, our interest is in 
	%evaluating
	%the effect of several testing procedures on gene expression correlation. 
	
	
	
	\subsection{General setup}\label{subsec:generalsetup}
	\textit{Correlation} is a statistical quantity used to assess a possible linear relationship 
	between two random variables or two sets of data sets. The degree of correlation is measured by 
	\textit{correlation coefficient}, a scaler taking values on the interval $[-1, 1]$. Correlation 
	coefficient of $+1$ ($-1$) indicates perfect positive (negative dependence), while correlation 
	coefficient of 0 implies no linear relationship between two random variables. Larger 
	correlation coefficient (in absolute value) corresponds to stronger linear correlation. 
		
	There are a number of ways to look at the correlation coefficient, many of which are special 
	cases of \textit{Pearson's correlation coefficient} 
	\citep{lee1988thirteen}. For example, the \textit{Kendall tau rank correlation coefficient} is 
	computed as Pearson's correlation coefficient between the ranked variables. Throughout this 
	paper, we will discuss Pearson's correlation under bivariate settings. We will restrict our 
	interest, following the notation of \citet{lee1988thirteen},  to two 
	types of Pearson's correlation coefficient. The first type of correlation, which we refer to as 
	\textit{\popucor}, is the standardized 
	covariance
	\begin{equation}\label{eq:popucor}
	\rho =\dfrac{\cov(X, Y)}{\sqrt{\var(X)\var(Y)}} = 
	\dfrac{E[(X-\mu_X)(Y_j-\mu_Y)]}{\sigma_X\sigma_Y}.
	\end{equation} 
	In equation (\ref{eq:popucor}), $\mu_X$ and $\mu_Y$ are the expected values of 
	random variables 
	$X$ and 
	$Y$,  and $\sigma_X<\infty$ and 
	$\sigma_Y<\infty$ are the population standard errors. The second type of correlation, which we 
	refer to as \textit{\samplecor}, is a function of raw scores and means
	\begin{equation}\label{eq:samplecor}
	r  =  \dfrac{\sum_j (x_j -\bar{x})(y_j - \bar{y})}{\sqrt{\sum_{j}(x_j - \bar{x})^2\sum_i(y_j - 
			\bar{y})^2}}, 
	\end{equation}
	where $(\bar{x}, \bar{y})$ is the vector of arithmetic mean of the observations. 
	\citet{fisher1915frequency} proved that sample correlation $r$ is a consistent 
	estimator for 
	population correlation $\rho$.
	
	Let $(X_j, Y_j)$ be a bivariate random variable representing two features (genes) of sample $j 
	= 1, \ldots, m$ , and $(x_j, y_j)$ the corresponding realization.
	We assume that the population mean of $(X_j, Y_j)$ may differ across samples, but that the 
	population covariance structure remains the 
	same, that is,  
	\begin{equation}\label{eq:meanstruct}
	E  \left(\begin{array}{c}
	X_j\\
	Y_j\\	
	\end{array} \right) 
	= 	\left(\begin{array}{c}
	\mu_{X,j}\\
	\mu_{Y,j}\\
	\end{array} \right)\stackrel{\text{def}}{=} \bm \mu_j,  \text{~~ for $j = 1, \ldots, m$}
	\end{equation}
	and 
	\begin{equation}\label{eq:covstruct}
	\cov\left(\begin{array}{c}
	X_j\\
	Y_j\\	
	\end{array} \right)	
	= \left(
	\begin{array}{cc}
	\sigma_X^2 &\rho \sigma_X\sigma_Y \\
	\rho \sigma_X \sigma_Y & 	\sigma_Y^2 \\
	\end{array} 
	\right)
	\stackrel{\text{def}}{=} \bm \Sigma 
	\end{equation}
	where $\rho$ is the population correlation defined by equation (\ref{eq:popucor}). In addition, 
	we assume independence across samples (note that independence implies 0 correlation, but not 
	vise versa), 
	\begin{equation}\label{eq:indepsamples}
	\cov(X_{j_1}, X_{j_2}) = \cov(Y_{j_1}, Y_{j_2}) = 0 \text{~~~for $j_1\neq j_2$}
	\end{equation}
	
	
	
	
	
	In the context of gene expression study, the goal is to detect DE---whether the expression 
	level of a gene is 
	significantly correlated with treatment or experimental variables. Let $\bm a:=(a_1, \ldots, 
	a_m)^T$ be a vector for a contrast of 
	interest, then DE detection for gene $X$ can be statistically formulated as 
	\begin{equation}\label{eq:hypotheses}
	H_{0}:  \bm a^T\bm \mu_X = 0 \textit{     Versus   }  H_{1}: \bm a^T\bm \mu_X \neq 0,
	\end{equation}
	where $\bm X = (X_1, \ldots, X_m)^T$ and $\bm \mu_{X} = (\mu_{X, 1}, \ldots, \mu_{X, m})^T$. 
	DE detection for gene $Y$ can be obtained by applying the same contrast to $\bm Y = (Y_1, 
	\ldots, Y_m)$ (simply replacing the subscript $X$ by $Y$ in equation (\ref{eq:hypotheses})).
	This hypothesis testing procedure usually results in a ``$t$-test similar" test statistic, in 
	which the numerator is a linear combination of 
	$\bm X$ and the denominator is its standard error. Without a loss of generality, we express the 
	test statistics as follows
	\begin{equation}\label{eq:teststat}
	T_X = \dfrac{\bm a^T\bm X}{S_X},  ~~~ T_Y = \dfrac{\bm a^T \bm Y}{S_Y},
	\end{equation}  
	where $S_X$ and $S_Y$ are the standard error for $\bm a^T\bm X$ and $\bm a^T\bm Y$ 
	respectively.	
	Our main goal is to explore the relationship between population correlation (equation 
	(\ref{eq:popucor})) for the test statistics 
	\begin{equation}
	\rho_T= \lim\limits_{m\rightarrow\infty}\rho_T(m) = \lim\limits_{m\rightarrow\infty}\cor({T_X, 
		T_Y}),
	\end{equation}  
	and that for their corresponding expression levels 
	\begin{equation}
	\rho = \cor(X, Y). 
	\end{equation}
	We will examine two typical test statistics having the form of equation (\ref{eq:teststat}).%, 
	%for a given sample size of $n_1$ for treatment group and $n_2$ for control group.
	
	
	
	\subsection{Results}\label{section:tcorresults}
	
	In this section we present the exact formula of test statistics correlation $\rho_T(m)$ by 
	making some assumptions about $T_X$ and $T_Y$, and show that the test statistics correlation 
	$\rho_T$ does not always equal the population correlation $\rho$. For the case of two-group 
	comparison, we prove that 1) if $T_X$ (or $T_Y$) is a linear transformation of $\bm X$ (or $\bm 
	Y$), 
	then $\rho_T= \rho$, and that 2) if $T_X$ (or $T_Y$) is the two sample $t$-test statistic for 
	$\bm 
	X$ (or $\bm Y$), then $|\rho_T| \leq |\rho|$. For 2), we show that the relationship between 
	$\rho_T$ and $\rho$ depends on whether the hypotheses tests (equation \ref{eq:hypotheses}) are 
	true null or not. We perform simulations for the case of test statistics derived from 
	two-sample $t$-test to illustrate our findings.
	
	%	 for two
	%	sample $t$-test. In the first part, we conclude theoretically that test statistics 
	%correlation 
	%and
	%	sample correlation are perfect positive dependent for two sample $z$-test, but that is not 
	%always
	%	true for two sample $t$-test. In the second part, we simulate four different cases where 
	%test
	%	statistics correlation $r_{\text{statistics}}$ may be very different from true correlation 
	%$\rho$ or
	%	sample correlation $r_{\text{sample}}$. 
	\subsubsection{Theory}
	%	\subsubsection{$S$ is a constant}
	
	\begin{theorem}\label{thm:teststatcor}
		Let $(X_j, Y_j), j = 1, \ldots, m$ be independent random vectors with mean and covariance 
		structures specified in equation (\ref{eq:meanstruct}). If $(\bm a^T\bm X, \bm a^T\bm Y)$ 
		is independent of $(S_X, S_Y)$, then the correlation of $T_X$ and $T_Y$ in equation 
		(\ref{eq:teststat}) can be expressed as 
		\begin{equation}\label{eq:teststatcor}
		\rho_T(m) = \frac{ \rho E(S_X^{-1}S_Y^{-1}) + \frac{\bm a^T\bm \mu_X\cdot \bm a^T\bm 
				\mu_Y}{\sigma_X\sigma_Y\bm a^T\bm a}\cov(S_X^{-1}, S_Y^{-1}) 
				}{\sqrt{\left[E(S_X^{-2}) 
				+ \frac{(\bm a^T\bm \mu_X)^2}{\sigma_X^2\bm a^T\bm 
					a}\var(S_X^{-1})\right]\left[E(S_Y^{-2}) + \frac{(\bm a^T\bm 
					\mu_Y)^2}{\sigma_Y^2\bm 
					a^T\bm a}\var(S_Y^{-1})\right]}}
		\end{equation}
		%	where $d_X = E(\bm a^T\bm \mu_X)$ and $d_Y = E(\bm a^T\bm Y)$.
	\end{theorem}
	\textbf{Proof:} Since samples are independent, we have 
	\begin{equation}\label{eq:testprepare}
	\begin{aligned}
	\cov(\bm a^T\bm X, \bm a^T\bm Y) &= \bm a^T \cov(\bm X, \bm Y)\bm a  = \rho\sigma_X\sigma_Y\bm 
	a^T\bm a, \\
	\var(\bm a^T\bm X)&  = \sigma_X^2\bm a^T\bm a, \\
	E(\bm a^T\bm X)^2& =(\bm a^T\bm \mu_X)^2 + \sigma_X^2\bm a^T\bm a, \\
	E[\bm (\bm a^T\bm X)(\bm a^T\bm Y)] &=E(\bm a^T\bm X)E(\bm a^T\bm Y) + \cov(\bm a^T\bm X, \bm 
	a^T\bm Y)  	\\& = (\bm a^T\bm \mu_X)(\bm a^T\bm \mu_Y) + \rho \sigma_X\sigma_Y\bm a^T\bm a
	\end{aligned}
	\end{equation}
	Note that since $S_X$ is independent of $S_X$, we have 
	\begin{equation}\label{eq:testdenom1}
	\begin{aligned}
	\var(T_X) &= E\left[\left(\frac{\bm a^T\bm X}{S_X}\right)^2\right] - \left[E\left(\frac{\bm 
	a^T\bm X}{S_X}\right)\right]^2\\
	& = E[\bm a^T\bm X]^2E[S_X^{-2}] - \left[E(\bm a^T\bm X)\right]^2\left[E(S_X^{-1})\right]^2\\
	& = \sigma_X^2\bm a^T\bm a E(S_X^{-2}) + (\bm a^T\bm \mu_X)^2\var(S_X^{-1})
	\end{aligned}
	\end{equation}
	Similarly, 
	\begin{equation}\label{eq:testdenom2}
	\var(T_Y)= \sigma_Y^2\bm a^T\bm a E(S_Y^{-2}) + (\bm a^T\bm \mu_Y)^2\var(S_Y^{-1})
	\end{equation}
	and
	\begin{equation}\label{eq:testnumerator}
	\begin{aligned}
	&\cov(T_X, T_Y) = E\left[\frac{ \bm a^T\bm X }{S_X^{-1}}\cdot\frac{ \bm a^T\bm Y 
	}{S_Y^{-1}}\right] - E\left[\frac{\bm a^T\bm X}{S_X^{-1}}\right]E\left[\frac{\bm a^T\bm 
	Y}{S_Y^{-1}}\right] \\
	& = E[\bm (\bm a^T\bm X)(\bm a^T\bm Y)]\cdot E[S_X^{-1}S_Y^{-1}]-(\bm a^T\bm \mu_X)(\bm a^T\bm 
	\mu_Y)E[S_X^{-1}]E[S_Y^{-1}]\\
	& = [(\bm a^T\bm \mu_X)(\bm a^T\bm \mu_Y)+ \rho \sigma_X\sigma_Y\bm a^T\bm 
	a]E[S_X^{-1}S_Y^{-1}]- (\bm a^T\bm \mu_X)(\bm a^T\bm \mu_Y)E[S_X^{-1}]E[S_Y^{-1}]
	\end{aligned}
	\end{equation}	
	The result follows by plugging equations (\ref{eq:testprepare})-(\ref{eq:testnumerator}) into 
	equation (\ref{eq:popucor}).
	
	\begin{corollary}\label{thm:lineartransformation} 
		For any non zero $\bm a$, $\rho_T(m)=\rho$ if $S_X$ and $S_Y$ are constant with respect to 
		$\bm X, \bm Y$. 
	\end{corollary}
	\textbf{Proof}: When $S_X$ and $S_Y$ are constants, $\cov(S_X^{-1}, S_Y^{-1})$, $\var(S_X^{-1}) 
	$ and $\var(S_Y^{-1})$ are all 0, and equation (\ref{eq:teststatcor}) reduces to 
	\begin{equation}
	\rho_T(m) = \frac{\rho E(S_X^{-1}S_Y^{-1})}{\sqrt{E(S_X^{-2})E(S_Y^{-2})}} = \rho.
	\end{equation}
	Corollary \ref{thm:lineartransformation} states that test statistics correlation and expression 
	level correlation are equal under linear transformation of $\bm X$ and $\bm Y$. 
	
	However, if we 
	assume that $(S_X, S_Y)$ is a non-constant function of $(\bm X, \bm Y)$, then the test 
	statistics correlation in equation (\ref{eq:teststatcor}) can be expressed as  
	\begin{equation}
	\rho_T(m) = \frac{ \frac{E(S_X^{-1}S_Y^{-1})}
		{\sqrt{\var(S_X^{-1})\var(S_Y^{-1})}}\rho + \frac{(\bm a^T\bm \mu_X)(\bm a^T\bm 
		\mu_Y)}{\sigma_X\sigma_Y\bm a^T\bm a} \rho_s	
	}{\sqrt{\left[ \frac{E(S_X^{-2})}{\var(S_X^{-1})} + \frac{(\bm a^T\bm \mu_X)^2}{\sigma_X^2\bm 
	a^T\bm a}\right]\left[ \frac{E(S_Y^{-2})}{\var(S_Y^{-1})} + \frac{(\bm a^T\bm 
	\mu_Y)^2}{\sigma_Y^2\bm a^T\bm a}\right]}} 
	\end{equation}
	where 
	\begin{equation}
	\rho_s = \frac{\cov(S_X^{-1},S_Y^{-1})}{\sqrt{\var(S_X^{-1})\var(S_Y^{-1})}}.
	\end{equation}
	The correlation between test statistics $\rho_T(m)$ depends on the form of test statistics, and 
	in general,  may not converge to the population correlation $\rho$. 
	
	\subsubsection{Application of Theorem \ref{thm:teststatcor} under normal distribution}
	
	Many gene expression experiments are done to compare expression levels under two-treatment 
	conditions. For the rest of this section, we discuss the relationship between $\rho_T$ and 
	$\rho$ under such setting.
	Let $n = n_1 + n_2$ be the total number of samples, where $n_1$ of them are from group 1 and 
	$n_2$ from group 2, and let
	\begin{equation}\label{eq:contrast}
	\bm a  = (\underbrace{\frac{1}{n_1}, \ldots, \frac{1}{n_1}}_{n_1}, \underbrace{-\frac{1}{n_2}, 
	\ldots, -\frac{1}{n_2}}_{n_2})^T
	\end{equation}
	be the contrast of interest. 
	The mean expression levels are specified as 
	\begin{equation}\label{eq:meanTwogroup}
	\begin{aligned}
	\bm \mu_j &= (\mu_X, \mu_Y)^T,~~ j = 1, \ldots, n_1, \\
	\bm \mu_j &= (\mu_X,  \mu_Y)^T  + ( \Delta_X,\Delta_Y)^T, ~~j = n_1 + 1, \ldots, n_1 + n_2.
	\end{aligned}
	\end{equation}
	
	
	If we set $S_X=1$, then $T_X$ corresponds to mean difference between groups 1 and 2; instead, 
	if $S_X = \sigma_X\sqrt{\frac{1}{n_1} + \frac{1}{n_2}}$ where $\sigma_X$ is known, then $T_X$ 
	corresponds to the statistic for two sample $z$-test. Therefore, according to Corollary 
	\ref{thm:lineartransformation},  $\rho_T=\rho$ if we use mean difference or $z$-value as test 
	statistics.
	
	The two sample $t$-statistic is also a commonly used statistic in differential expression 
	analysis.	In the case of two sample $t$-test with equal variance, with the contrast $\bm a$ 
	defined in equation (\ref{eq:contrast}), the test statistic for $X$ is 
	\begin{equation}\label{eq:tx}
	T_X= \frac{\bar{X}_1- \bar{X}_2}{S_{p, X}\sqrt{\frac{1}{n_1} + \frac{1}{n_2}}},
	\end{equation}
	where $S_{p, X}$ is the pooled variance
	\begin{equation}\label{eq:tstatform}
	\begin{aligned}
	&S_{p, X}^2 = \frac{(n_1-1)S_{X, 1}^2 + (n_2 -1)S_{X,2}^2}{n_1 + n_2 -2}. \\
	% 	&S_{X, 1}^2 =  \frac{\sum_{j=1}^{n_1}(X_j - \bar{X}_1)^2}{n_1 -1}, ~~S_{X, 2}^2 =  
	%\frac{\sum_{j=n_1 +1}^{n_1 + n_2}(X_j - \bar{X}_2)^2}{n_2 -1},\\
	\end{aligned}
	\end{equation}
	Similarly, we obtain $T_Y$ by replacing the subscript ``$X$" in equations (\ref{eq:tx}) and 
	(\ref{eq:tstatform}). Under normal distribution assumption, we have the following theorem for 
	two sample $t$-test with equal variance:
	\begin{theorem}\label{thm:tstat}
		Let $(X_i, Y_i), i = 1, \ldots, n$ follow a bivariate normal distribution with mean 
		specified by equations (\ref{eq:meanTwogroup}) and covariance $\bm \Sigma$ (see equation 
		(\ref{eq:meanstruct})). If $T_X$ and $T_Y$ are statistics for equal-variance two-sample 
		$t$-test, then 
		\begin{equation}\label{eq:ttestcor}
		\begin{aligned}
		\cor(T_X, T_Y) =   
		\frac{\frac{\Delta_X\Delta_Y}{\sigma_X\sigma_Y}C \rho_{s}+ \rho B
			+ \rho_{s}\rho(A-B)}{\sqrt{\left[ \frac{\Delta_X^2}{\sigma_X^2}C + 
			A\right]\left[\frac{\Delta_Y^2}{\sigma_X^2}C +   A\right]}}
		\end{aligned}
		\end{equation}
		where 
		\begin{equation}\label{eq:AandB}
		\begin{aligned}
		A & = \frac{n_1 + n_2-2}{n_1 + n_2-4}, ~~B =
		\frac{(\frac{n_1 + n_2 -2}{2})\Gamma^2(\frac{n_1 + n_2 -4}{2} + 
		\frac{1}{2})}{\Gamma^2(\frac{n_1+ n_2 -2}{2})}, \\
		\rho_s & = \cor(S_X^{-1}, S_Y^{-1}), ~~ 
		C = \frac{(n_1 + n_2)(A-B)}{(2 + n_1n_2^{-1} + n_1n_2^{-1})}.
		\end{aligned}
		\end{equation}	 
	\end{theorem}
	The proof of Theorem \ref{thm:tstat} is presented in Section \ref{section:testcormethod}. Next 
	we present the limit of $\cor(T_X, T_Y)$.
	\begin{theorem}\label{thm:rholimit}
		If there exists positive constants $M_1$ and $M_2$, such that $M_1 \leq n_1n_2^{-1}\leq 
		M_2$, then
		\begin{equation}\label{eq:limitT}
		\rho_T=\lim\limits_{n_1 + n_2 \rightarrow \infty} \cor(T_X, T_Y) = \frac{\rho(1  +
			\beta\frac{\Delta_X\Delta_Y}{\sigma_X\sigma_Y}\rho)}{\sqrt{  \left[ 1 
			+\beta\frac{\Delta_X^2}{\sigma_X^2}\right]\left[ 1 + 
			\beta\frac{\Delta_Y^2}{\sigma_Y^2}\right]}}
		\end{equation}
		where %$\rho_{s}$ is defined in equation (\ref{eq:AandB}) and 
		$\beta = \lim\limits_{n_1 + n_2 \rightarrow \infty}C = (4 + 2n_1^{-1}n_2 + 
		2n_1n_2^{-1})^{-1}$.
	\end{theorem}
	Theorem \ref{thm:rholimit} says that as long as $n_1$ and $n_2$ grow proportionally to 
	infinity, the quantity $\rho_T$ is a function of population correlation $\rho$, the 
	signal-to-noise ratios $(\delta_X, \delta_Y) = (\Delta_X/\sigma_X, \Delta_Y/\sigma_Y)$  and the 
	sample ratio $n_1/n_2$. 
	We have the following observations:
	\begin{enumerate}
		\item If both tests are true null (i.e., $\bm \Delta = \bm 0$), then $\rho_T = \rho$.
		\item If only one test is true null, then $\rho_T$ is proportional to and smaller in 
		absolute value than $\rho$ (i.e., $\rho_T = \gamma_0\rho,~ 0 <\gamma_0 <1$).
		\item If both tests are true alternative (i.e., $\bm \Delta \neq \bm 0$), then $\rho_T\neq 
		\rho$ in general. Specifically,
		\begin{enumerate}
			\item[i)]  when $\Delta_X\Delta_Y >0$ (i.e., both genes are DE towards the same 
			direction), we have $\rho_T>\rho$ for $\rho <0$ and $0 \leq \rho_T \leq\rho$ for $\rho 
			\geq 0$.
			\item[ii)] when $\Delta_X\Delta_Y <0$ (i.e., genes are DE towards different 
			directions), we have
			$\rho <\rho_T<0$ for $\rho <0$ and $\rho_T<\rho$ for $\rho>0$.
		\end{enumerate}
		Therefore in either case, we have $|\rho_T| \leq |\rho|$. 
	\end{enumerate}
	
	
	
	%			\begin{figure}[!ht]
	%				\centering
	%				\includegraphics[width=6.5cm, height= 6.5cm]{Figures/th1.eps}
	%				\includegraphics[width=6.5cm, height= 6.5cm]{Figures/th2.eps}
	%				\includegraphics[width=6.5cm, height= 6.5cm]{Figures/th3.eps}
	%				\includegraphics[width=6.5cm, height= 6.5cm]{Figures/th4.eps}
	%				\caption{Theoretical correlation between test statistics under different 
	%settings. 
	%				For each of the fixed correlation $\rho$, we adjust delta1 
	%($\Delta_x/\sigma_x$) 
	%				and delta2 ($\Delta_y/\sigma_y$) to evaluate $\rho_T$ according to equation 
	%				(\ref{eq:limitT}).}
	%				\label{fig:th}
	%			\end{figure}
	%			
	
	We note that $|\rho_T| \leq |\rho|$ when test statistics are derived from two sample $t$ test 
	with equal variance. In other words, $T_X$ and $T_Y$ are always ``no more correlated" than $X$ 
	and $Y$ are. It's also interesting to note that when both genes are DE, $\rho_T=0$ at $\rho 
	=-\frac{\sigma_X\sigma_Y}{\beta\Delta_X\Delta_Y} $ and 
	$\frac{\sigma_X\sigma_Y}{\beta\Delta_X\Delta_Y} \in (-1, 1)$. Figure \ref{fig:ct} shows the 
	contour plots of $\rho_T$ versus the signal-to-noise ratios $\delta_X$ ($=\Delta_X/\sigma_X$) 
	and $\delta_Y$ ($=\Delta_Y/\sigma_Y$) for different $\rho$'s. The largest value of $\rho_T$ (in 
	absolute value) 
	is always at the center, where both $\delta_X$ and $\delta_Y$ are 0 (i.e., $\Delta_X =\Delta_Y 
	= 0$).
	
	\begin{figure}[!ht]
		\centering
		\includegraphics[width=6.5cm, height= 6.5cm]{Figures/ct1.eps}
		\includegraphics[width=6.5cm, height= 6.5cm]{Figures/ct2.eps}
		\includegraphics[width=6.5cm, height= 6.5cm]{Figures/ct3.eps}
		\includegraphics[width=6.5cm, height= 6.5cm]{Figures/ct4.eps}
		\caption[Contour plot of theoretical correlation between test statistics.]{Contour plot of 
		theoretical correlation between test statistics. For 
			each fixed $\rho$ and each pair of $\delta_X$ (	$=\Delta_X/\sigma_X$) 
			and $\delta_Y$ ($=\Delta_Y/\sigma_Y$), the theoretical correlation $\rho_T$ is 
			calculated according to equation (\ref{eq:limitT}).}
		\label{fig:ct}
	\end{figure}
	
	In addition, if $n_1/n_2 \rightarrow 0$ or $\infty$, then $\beta = 0$ and we have $\rho_T = 
	\rho$. That is, when sample size of one group is not proportional to that of the other, 
	$\cor(T_X, T_Y)$ will converge to $\rho$ regardless of whether the tests are under the null or 
	not. 	
	%	 When $\bm \Delta = \bm 0$ then $\rho_T = \rho$; but when $\bm \Delta \neq \bm 0$, then 
	%$\rho_T \neq \rho$ in general. % In the next section, we will discuss about it in further 
	%%detail. 
	%	\begin{corollary} 
	%	If $\bm \Delta  = (\Delta_X, \Delta_Y)= \bm 0$, and there exists a positive number $M$, 
	%such that  $n_1n_2^{-1}\leq M$ and $n_1n_2^{-1}\leq M$,  then $\cor({T_X, T_Y})\rightarrow 
	%\rho$ as $n_1 + n_2 \rightarrow \infty$. 
	%	\end{corollary}
	%	Proof:  If null is true for both test or $\bm \Delta  = 0$, then equation 
	%(\ref{eq:ttestcor}) reduces to
	%	\begin{align}\label{CalculateTCor}
	%		\cor(T_X, T_Y) = \left[\rho_s \cdot 1 + (1-\rho_s)\frac{B}{A}\right]\rho
	%	\end{align}
	%	The term in the square bracket is a weighted average of 1 and $\frac{B}{A}$, with the latter
	%	converging to 1 as $n_1 + n_2$ grows to infinity. Therefore $\lim\limits_{n_1 + 
	%n_2\rightarrow\infty} \cor({T_X, T_Y}) = \rho$.
	%	\begin{corollary} 
	%		If $\bm \Delta = (\Delta_X, \Delta_Y)\neq \bm 0$, then  $\cor({T_X,
	%			T_Y})$ does not converge to $\rho$ in general.
	%	\end{corollary} 
	%	The result immediately follows from lemma  (\ref{lemmaLimit}) in appendix.
	%	Depending on the underlying value of $\bm\Delta$ (DE or not DE, up-regulated or 
	%down-regulated if DE) and covariance $\bm \Sigma$,
	%	$\rho_T$ might be far from $\rho$ in different ways. 
	
	%	In the Methods section, Lemma \ref{thm:invScorlimit} states that
	
	
	\subsubsection{Simulation}
	We perform simulations to evaluate the correlations between test statistics and those between 
	expression levels under two sample $t$-test. We simulate the expression data from normal 
	distributions. Specifically, we let $(X, Y)$ be the expression levels of genes $X$ and $Y$, and
	\begin{equation}
	\begin{aligned}
	&\left( \begin{array}{c}
	X_{j}\\
	Y_{j}\\
	\end{array}\right)
	\sim N\left[
	\left(\begin{array}{c}
	0\\
	0\\
	\end{array} \right), 
	\left(
	\begin{array}{cc}
	1 &\rho  \\
	\rho & 	1 \\
	\end{array}
	\right)
	\right], j = 1, \ldots, n_1 \\
	& \left( \begin{array}{c}
	X_{j}\\
	Y_{j}\\
	\end{array}\right)
	\sim N\left[
	\left(\begin{array}{c}
	\Delta_X\\
	\Delta_Y\\
	\end{array} \right), 
	\left(
	\begin{array}{cc}
	1 &\rho \\
	\rho  & 	1 \\
	\end{array}
	\right)
	\right], j = n_1 +1, \ldots, n_1 + n_2 
	\end{aligned}
	\end{equation}
	%where $j_1 = 1,\ldots, n_1$ and $j_2 = n_1 + 1, \ldots, n_1 + n_2$.
	For each given $\rho$, we 
	consider these $n=n_1 + n_2$ pairs of $(X, Y)$
	as observations from one \textit{simulated} experiment. Out of this experiment, we calculate $q 
	= (T_X, T_Y)$ where $T_X$ and $T_Y$ are the test statistics for gene $X$ and gene $Y$ 
	respectively using 
	two-sample $t$-test for equal variance procedure. We replicate the simulated
	experiment for $B=1000$ times, resulting in a matrix $\bm Q_{1000\times 2}$. We take the 
	correlation between the first and the second columns of $\bm Q$ as an estimate for test 
	statistics correlation	$r_\text{statistics}$. 
	%	Sample correlation is a consistent estimator for underlying true correlation, therefore
	%		$r_\text{statistics}$ and $r_{\text{sample}}$ should reflect the true correlation 
	%between $	
	%%% 		T_X$ and $T_Y$ and 
	%		that between $X$ and $Y$ respectively. 
	We increase $\rho$ from $-0.99$ to $0.99$ by fixed step size $0.01$, and examine the 
	relationship between $r_\text{statistics}$ and $\rho$ under the following different cases:
	\begin{enumerate}
		\item[a)]  $\delta_X = \delta_Y  =0$;
		\item[b)]  $\delta_X = 0, \delta_Y=2$;
		\item[c)]  $\delta_X = 0.5, \delta_Y=2$;
		\item[d)]  $\delta_X = 1, \delta_Y=2$;
		\item[e)]  $\delta_X = 3, \delta_Y=2$;
		\item[f)]  $\delta_X = -3, \delta_Y=2$.
	\end{enumerate}
	
	We conduct simulations for two different sample sizes: we set $n_1 = n_2 = 1000$ to assess 
	asymptotic performance of $\rho_T(n)$ in equation (\ref{eq:teststatcor}), and set $n_1 = n_2 
	= 3$ to mimic small sample size scenarios which are typical in gene expression study. 
	
	In Figure \ref{fig:tstat}, we plot $r_\text{statistics}$ against the 
	underlying true population correlation $\rho$ under both large and small sample size scenarios. 
	In case a) where both tests are true null, $r_\text{statistics}$ is close to the true 
	correlation $\rho$ when sample size is large ($n_1 = n_2 = 1000$), but smaller (in absolute 
	value) than $\rho$ when sample size is small ($n_1 = n_2 = 3$).
	In cases b)---f) where there is at least one true alternative, the estimate
	$r_\text{statistics}$ can be very 
	different from $\rho$. In case b) where only one gene is DE, 
	the magnitude of $r_\text{statistics}$ is proportional to, and smaller in absolute value than 
	$\rho$.
	% and according to equation (\ref{eq:ttestcor}), the magnitude of $r_\text{statistics}$ has to 
	%do with the signal-to-noise ratio $\Delta/\sigma$
	It is more interesting to note that $r_\text{statistics}$ is not monotone with respect to 
	$\rho$ when both genes are DE. If genes are DE towards the same direction as in the case of 
	e),  $r_\text{statistics}$ first decreases until it reaches the minimum (a negative value), and 
	then gradually increases to 1, as 
	$\rho$ grows from $-1$ to $1$. When genes are DE towards opposite directions like the case f), 
	however, the trend is reversed from that of e): $r_\text{statistics}$ increases from $-1$ to 
	its maximum (a positive value), and then decreases. 
	This set of simulation results is reflected in the test statistics correlation formula of 
	equation (\ref{eq:limitT}). We also demonstrate the process of how $\rho_T$ changes from
	being a linear function of $\rho$ to a quadratic function, by fixing $\delta_Y=2$ while 
	increasing $\delta_X$ from $0$ (case b) to $3$ (case e).
	
	We illustrate in Figure \ref{fig:tstat} the variation in $r_\text{statistics}$ with 
	respect to change in sample size $n$. For each fixed $\rho$ under cases a)---f), the absolute 
	value of $r_\text{statistics}$  increases when we change the sample size $n$ from $6$ to 
	$2000$. The change in $r_\text{statistics}$ induced by sample size could be substantial, 
	especially when the population correlation is large (e.g., $\rho > 0.2$). This simulation shows 
	that test statistics correlation $\rho_T$ can be over-estimated by sample correlation, 
	especially when sample size is small.
	%	We note that in all of the four cases, the inequality $r_\text{statistics}\leq \rho$ holds.
	
	\begin{figure}[!th]
		\centering
		\includegraphics[width=4.8cm, height= 6.2cm]{Figures/case1.eps}
		\includegraphics[width=4.8cm, height= 6.2cm]{Figures/case2.eps}
		\includegraphics[width=4.8cm, height= 6.2cm]{Figures/case3.eps}
		\includegraphics[width=4.8cm, height= 6.2cm]{Figures/case4.eps}
		\includegraphics[width=4.8cm, height= 6.2cm]{Figures/case5.eps}
		\includegraphics[width=4.8cm, height= 6.2cm]{Figures/case6.eps}
		\caption[Plots of test statistics correlation against true population 
		correlation]{Plots of test statistics correlation against true population 
			correlation. The test statistics are calculated 
			using two sample $t$-test with equal variance, and the theoretical correlation is 
			calculated by 
			equation (\ref{eq:limitT}).}
		\label{fig:tstat}
	\end{figure}
	
	
	
	
	
	
	\subsection{Method}\label{section:testcormethod}
	
	\begin{lemma}
		The sample correlation coefficient $r$ defined in equation (\ref{eq:samplecor}) is a 
		consistent estimator for the population correlation $\rho$, 
		\[\sqrt{n}(r - \rho ) \stackrel{D}{\rightarrow}N\left(0, (1-\rho^2)^2\right).\]
	\end{lemma}
	The proof of Lemma 1 can be found in \citet{fisher1915frequency}. \\
	%$\rho({G_1, G_2})$ and $\frac{{\rho}_X + {\rho}_Y}{2}$ are asymptotically equivalent. \\
	
	To prove Theorem \ref{thm:tstat}, it is useful to note that $\bm U = (\bm a^T\bm X, \bm a^T\bm 
	Y)$ is independent of $\bm S = (S_X, S_Y)$,
	following from Lemmas \ref{lemmabiChisq} and \ref{lemmaIndep}.
	\begin{lemma}\label{lemmabiChisq}
		Let $(X_{j}, Y_{j}), j=1 \ldots,  m$ be independent random variables satisfying equation 
		(\ref{eq:indepsamples}),
		then $\bm W = (W_{X},W_{Y}) =(\frac{(m -1)S_{X}^2}{\sigma_X^2}, 
		\frac{(n-1)S_{Y}^2}{\sigma_Y^2})$ 
		follows a \textbf{bivariate chi square distribution} with density 
		\begin{equation}\label{biChisq}
		\begin{aligned}
		f(w_x, w_y) & = \frac{2^{-m}(w_xw_y)^{(n-3)/2}e^{-\frac{w_x +
					w_y}{2(1-\rho^2)}}}{\sqrt{\pi}\Gamma(\frac{m}{2})(1-\rho^2)^{(m-1)/2}} 
		\times \\
		& \sum_{k=0}^{\infty}[1 +
		(-1)^k]\left(\frac{\rho\sqrt{w_xw_y}}{1-\rho^2}\right)^k\frac{\Gamma(\frac{k+1}{2})}{k!\Gamma(\frac{k+
				m}{2})}
		\end{aligned}
		\end{equation}
		for $n>3$ and $-1<\rho < 1$.
	\end{lemma}
	For proof of Lemma \ref{lemmabiChisq}, interested readers are referred to 
	\citet{joarder2009moments}.
	It immediately follows from Lemma \ref{lemmabiChisq} that $\bm W_1 = (\frac{(n_1 -1)S_{X, 
	1}^2}{\sigma_X^2}, \frac{(n_1-1)S_{Y, 1}^2}{\sigma_Y^2})$ follows bivariate chi-square 
	distribution with degree of freedom $n_1-1$. Similarly, $\bm W_2 =(\frac{(n_2 -1)S_{X, 
	2}^2}{\sigma_X^2}, \frac{(n_2-1)S_{Y, 2}^2}{\sigma_Y^2})$ follows a bivariate chi-square 
	distribution with degree of freedom $n_2-1$.  Note that $\bm W_1$ and $\bm W_2$ are independent 
	since the samples are independent. 
	
	\begin{lemma}\label{lemmaIndep}
		$\bm U =(U_X, U_Y)$ is independent of $\bm S = (S_X ,S_Y)$.
	\end{lemma}
	\textbf{Proof}: By Lemma \ref{lemmabiChisq}, the density function of $
	\bm W_1 + \bm W_2$ only involves $\sigma^2_X, \sigma^2_Y, \rho$ and sample size $n_1, n_2$, 
	therefore
	we can denote its density by some function $g(\sigma^2_X, \sigma^2_Y, \rho,
	n_1 + n_2)$. Note that $\bm S^2 = \frac{(\sigma_X^2, ~\sigma^2_Y)}{n_1 +n_2 -2}(\bm W_1 + \bm 
	W_2)^T $
	is a linear transformation of $\bm W_1 + \bm W_2$, so its density also can be expressed in 
	terms of $\sigma^2_1, \sigma^2_2, \rho, n_1, n_2$. Therefore $\bm S = (S_X ,S_Y)$ is an 
	ancillary statistic for $\bm \Delta$. On the other hand, it can
	be shown that $\bm U =(U_X, U_Y)$ is a complete sufficient statistic for $\bm \Delta$. It 
	follows by
	Basu's theorem that $\bm U$ and $\bm S$ are independent. 
	
	
	Lemma \ref{lemmaIndep} implies that  $U_XU_Y$ is also independent of $S_X^{-1}S_Y^{-1}$, and
	therefore $E(\frac{U_X}{S_X} \cdot\frac{U_Y}{S_Y})$ can be expressed as
	$E(U_XU_Y)E(S_X^{-1}S_Y^{-1})$. We can apply Theorem \ref{thm:teststatcor} to calculate the 
	correlation between $T_X$ and $T_Y$ under two sample $t$-test for equal variance. 
	
	\textbf{Proof of theorem \ref{thm:tstat}} \\
	First note that by Lemma  \ref{lemmaIndep} we have
	\begin{align*}
	\cov(T_X, T_Y) &= E(T_XT_Y) - E(T_X)E(T_Y) \\
	%& = E(c_0\frac{U_1}{S_1} \cdot c_0\frac{U_2}{S_2}) - E(c_0\frac{U_1}{S_1})E( 
	%c_0\frac{U_2}{S_2}) \\
	& = \frac{1}{c_0^2} \left[E(U_XU_Y)E(S_X^{-1}S_Y^{-1}) - E(\frac{U_X}{S_X})E( 
	\frac{U_Y}{S_Y})\right]   
	\end{align*}
	where $c_0 = \sqrt{\frac{1}{n_1} + \frac{1}{n_2}}$ and $\var(T_X) = \var(\frac{U_X}{c_0S_X})=
	\frac{1}{c_0^2}\var(\frac{U_X}{S_X})$. 
	Note that 
	\begin{equation}\label{eq:Tcorrelation}
	\begin{aligned}
	\cor(T_X, T_Y) & = \frac{\cov(T_X, T_Y) }{\sqrt{\var(T_X) \var(T_Y) }} \\
	& = \frac{E(U_XU_Y)E(S_X^{-1}S_Y^{-1}) - E(\frac{U_X}{S_X})E(
		\frac{U_Y}{S_Y})}{\sqrt{\var(\frac{U_X}{S_X})\var(\frac{U_Y}{S_Y})}} 
	\end{aligned}
	\end{equation}
	We need to calculate $E(U_XU_Y)$, $E(S_X^{-1}S_Y^{-1})$, $ E(\frac{U_i}{S_i})$ and
	$\var(\frac{U_i}{S_i})$ for $i =X, Y$. 
	\begin{enumerate}
		\item Note that $U_i\sim N\left(\Delta_i, \sigma_i^2(\frac{1}{n_1} + \frac{1}{n_2})\right), 
		i=X, Y$. 
		\begin{equation}\label{eq1}
		\begin{aligned}
		E(U_XU_X)&= \cov(U_X, U_Y) + E(U_X)E(U_Y) \\
		&= \rho\sigma_X\sigma_Y\left(\frac{1}{n_1} + \frac{1}{n_2}\right) +
		\Delta_X\Delta_Y
		\end{aligned} 
		\end{equation}
		
		\item Since $\frac{(n_1-1)S_{X}^2}{\sigma_X^2}$ and $\frac{(n_2-1)S_{Y}^2}{\sigma_Y^2}$ are
		independent and follow $\chi^2(n_1-1)$ and $\chi^2(n_2 -1)$ respectively, , we have 
		$W_{X}=\frac{(n_1 + n_2 -2)S_X^2}{\sigma_X^2}\sim
		\chi^2(n_1 + n_2-2)$. It can be shown that 
		\[E(W_{X}^k)= \frac{2^k\Gamma(\frac{n_1 + n_2 -2}{2}+k)}{\Gamma(\frac{n_1 + n_2 -2}{2})}\] 
		Therefore 
		\begin{equation}
		\begin{aligned}
		E\left(S_X^{-1}\right) =
		\frac{\sqrt{B}}{\sigma_X},		~~~\var\left(S_X^{-1}\right) = \frac{A-B}{\sigma_X^2}
		\end{aligned}
		\end{equation}
		Note that $\rho_s = \cor(S_X^{-1}, S_Y^{-1})$, we have 
		\begin{equation}\label{eq2}
		\begin{aligned}
		E(S_X^{-1}S_Y^{-1})  &= E(S_X^{-1})E(S_Y^{-1}) + \rho_s
		\sqrt{\var(S_X^{-1})\var(S_Y^{-1})} \\
		& = \frac{B}{\sigma_X\sigma_Y} + \rho_s
		\frac{A-B}{\sigma_X\sigma_Y}
		\end{aligned}
		\end{equation}
		
		\item $U_i\sim N\left(\Delta_i, \sigma_i^2(\frac{1}{n_1} + \frac{1}{n_2})\right)$ and 
		$\frac{(n_1 + n_2 -2)S_i^2}{\sigma_i^2} \sim
		\chi^2(n_1 + n_2-2)$ and by Lemma \ref{lemmaIndep}  $U_i$ and $\frac{(n_1 + n_2 
		-2)S_i^2}{\sigma_i^2}$ are independent for $i = X, Y$, we have 
		\begin{equation}
		\frac{\frac{U_i-\Delta_i}{\sigma_i\sqrt{\frac{1}{n_1} + \frac{1}{n_2}}}}{\frac{(n_1 + 
		n_2-2)S_i^2}{\sigma_i^2}/(n_1 + n_2 -2)}  =
		\frac{U_i-\Delta_i}{S_i\sqrt{\frac{1}{n_1 } + \frac{1}{n_2}}}\sim t(n_1 + n_2-2)
		\end{equation}	
		It follows from 
		\begin{equation}
		E\left(\frac{U_i-\Delta_i}{S_i\sqrt{\frac{1}{n_1} + \frac{1}{n_2}}}\right)=0, ~~ 
		\text{Var}\left(\frac{U_i-\Delta_i}{S_i\sqrt{\frac{1}{n_1} + \frac{1}{n_2}}}\right) = 
		\frac{n_1 + n_2-2}{n_1 + n_2-4}
		\end{equation}
		that
		\begin{equation}\label{eq3}
		\begin{aligned}
		E\left(\frac{U_i}{S_i}\right) &= \frac{\Delta_i}{\sigma_i}\sqrt{B} \\
		\var\left(\frac{U_i}{S_i}\right)&=A\left(\frac{1}{n_1} + \frac{1}{n_2}\right) + 
		\frac{\Delta_i^2}{\sigma_i^2}(A-B)
		\end{aligned}
		\end{equation}
	\end{enumerate}
	Finally,  the test statistics correlation (\ref{eq:ttestcor}) is obtained by plugging
	equations (\ref{eq1}--\ref{eq3}) into equation (\ref{eq:Tcorrelation}).
	
	\begin{lemma}\label{lemmaLimit}
		If there exists a positive number $M$, such that  $n_1n_2^{-1}\leq M$ and $n_1n_2^{-1}\leq 
		M$, then the following results hold:
		\begin{enumerate}
			\item $\lim\limits_{n_1 + n_2\rightarrow \infty} A = 1$.
			\item $\lim\limits_{n_1 + n_2\rightarrow \infty} B = 1$.
			\item $\lim\limits_{n_1 + n_2\rightarrow \infty} C = \beta.$
		\end{enumerate}
		where  $A, B$ and $C$ are defined in equation (\ref{eq:AandB}), and $\beta= (4 + 
		n_1n_2^{-1} + n_1^{-1}n_2)^{-1}$. 
	\end{lemma}
	\textbf{Proof}: Note that 
	\begin{equation}
	B = 
	\begin{cases}
	\frac{(k-1)\Gamma^2(k- \frac{3}{2})}{\Gamma^2(k-1)},& \text{if } n_1 + n_2 = 2k, k\geq 2 \\
	~\\
	\frac{(k-\frac{1}{2})\Gamma^2(k- 1)}{\Gamma^2(k-\frac{1}{2})},& \text{if } n_1 + n_2 = 2k+1, 
	k\geq 2 \\
	\end{cases}
	\end{equation}
	We will use second order Stirling's formula,
	\begin{align}\label{Stirling1}
	k! \approx \sqrt{2\pi k}\left(\frac{k}{e}\right)^k(1 + \frac{1}{12k})
	\end{align}
	Using Stirling's formula (\ref{Stirling1}) and  $\Gamma(k + \frac{1}{2}) =
	\frac{(2k)!}{4^kk!}\sqrt{\pi}$, it can be shown that 
	\begin{equation}\label{eq:Bapprox}
	B \approx  
	\begin{cases}
	\frac{(k-1)(k-2)(k-2 + \frac{1}{24})^2}{(k-2 + \frac{1}{12})^4},& \text{if } n_1 + n_2 = 2k, 
	k\geq 2 \\
	~\\
	\frac{(k-\frac{1}{2})(k - 1 + \frac{1}{12})^4}{(k-1+ \frac{1}{24})^2(k-1)^3},& \text{if } n_1 + 
	n_2 = 2k+1, k\geq 2 \\
	\end{cases}
	\end{equation}
	It can also be shown following equation (\ref{eq:Bapprox}) that
	\begin{equation}\label{eq:AminusB}
	A- B \approx  
	\begin{cases}
	\frac{\frac{1}{4}(k-1)(k-2)^3 + o((k-2)^4)}{(k-2)(k-2 + \frac{1}{12})^4},& \text{if } n_1 + n_2 
	= 2k, k\geq 2 \\
	~\\
	\frac{\frac{1}{4}(k-1)^3(k-\frac{1}{2})(k-3) + o((k-1)^4)}{(k-\frac{3}{2})(k-1+ 
	\frac{1}{24})^2(k-1)^3},& \text{if } n_1 + n_2 = 2k+1, k\geq 2 \\
	\end{cases}
	\end{equation}
	And the results immediately follow.
	
	\begin{lemma}\label{thm:invScorlimit}
		Let $(X_j, Y_j), j = 1, \ldots, n$ be i.i.d. random variables under the two sample $t$-test 
		for equal variance setting, 
		%	  \begin{equation}\notag
		%	   Z_j \sim N\left[ \left(
		%	   \begin{array}{c}
		%	     \mu_X\\
		%	     \mu_Y \\
		%	   \end{array} \right), 
		%	   \left(
		%	   \begin{array}{cc}
		%	   \sigma_X^2	& \rho\sigma_{X}\sigma_Y\\
		%	   	  \rho\sigma_{X}\sigma_Y & \sigma_Y^2\\
		%	   	  \end{array}
		%	   \right)
		%	   \right] 
		%	  \end{equation}
		with mean specified in equation (\ref{eq:meanTwogroup}) covariance structure in equation 
		(\ref{eq:meanstruct}). Then we have
		$\lim\limits_{n\rightarrow\infty}\rho_s = \rho^2$.
	\end{lemma}
	\textbf{Proof}: Let's first look at samples $j=1, \ldots, n_1$. Note that 
	\begin{equation}
	S_{X,1}^2= \frac{1}{n_1}\sum_{j=1}^{n_1}(X_j -\bar{X}_1)^2
	\end{equation}
	is the \textit{maximum likelihood estimator} (MLE) for $\sigma_X^2$. By invariance property of 
	MLE,
	%	 	\begin{equation}\label{eq:sampVarasypm}
	%	 	\begin{aligned}
	%	 	& \sqrt{n_1}\left[\left( \begin{array}{c}
	%	 	S_{X, 1}^2\\
	%	 	S_{Y, 1}^2\\
	%	 	\end{array}\right)
	%	 	-
	%	 	\left( \begin{array}{c}
	%	 	\sigma_X^2\\
	%	 	\sigma_Y^2\\
	%	 	\end{array}\right)
	%	 	\right]
	%	 	\stackrel{d.}{\longrightarrow} 
	%	 	N\left[
	%	 	\left(\begin{array}{c}
	%	 	0\\
	%	 	0\\
	%	 	\end{array} \right), 
	%	 	2\left(
	%	 	\begin{array}{cc}
	%	 	\sigma_X^4 &\rho^2\sigma_X^2\sigma_Y^2 \\
	%	 	\rho^2\sigma_X^2\sigma_Y^2  &\sigma_Y^4 \\
	%	 	\end{array}
	%	 	\right)
	%	 	\right] \\
	%	 		& \sqrt{n_2}\left[\left( \begin{array}{c}
	%	 		S_{X, 2}^2\\
	%	 		S_{Y, 2}^2\\
	%	 		\end{array}\right)
	%	 		-
	%	 		\left( \begin{array}{c}
	%	 		\sigma_X^2\\
	%	 		\sigma_Y^2\\
	%	 		\end{array}\right)
	%	 		\right]
	%	 		\stackrel{d.}{\longrightarrow} 
	%	 		N\left[
	%	 		\left(\begin{array}{c}
	%	 		0\\
	%	 		0\\
	%	 		\end{array} \right), 
	%	 		2\left(
	%	 		\begin{array}{cc}
	%	 		\sigma_X^4 &\rho^2\sigma_X^2\sigma_Y^2 \\
	%	 		\rho^2\sigma_X^2\sigma_Y^2  &\sigma_Y^4 \\
	%	 		\end{array}
	%	 		\right)
	%	 		\right] 
	%	 	\end{aligned}
	%	 	\end{equation}
	the pooled variance estimator 
	\begin{equation}\label{eq:Slinearcomb}
	\begin{aligned}
	&\left( \begin{array}{c}
	S_{X}^2\\
	S_{Y}^2\\
	\end{array}\right)
	= 
	a_1\left( \begin{array}{c}
	S_{X, 1}^2\\
	S_{Y,1}^2\\
	\end{array}\right)
	+
	a_2\left( \begin{array}{c}
	S_{X, 2}^2\\
	S_{Y, 2}^2\\
	\end{array}\right)
	\end{aligned}
	\end{equation}
	where 
	\[ 	n = n_1 + n_2, ~~a_1 = \frac{n_1 -1}{n-2}, ~~a_2 = \frac{n_2 -1}{n-2} \]
	is also MLE for $(\sigma_X^2, \sigma_Y^2)^T$ respectively.
	It can be shown that 
	\begin{equation}\label{eq:sampVarasypm}
	\begin{aligned}
	E[S_{X}^2] = \sigma_X^2,&~E[S_{Y}^2] = \sigma_Y^2,  \\
	\var[S_{X}^2] \rightarrow \frac{2\sigma_X^4}{n},~\var[S_{Y}^2] \rightarrow 
	\frac{2\sigma_Y^4}{n},&
	~\cov(S_{X}^2, S_{Y}^2) \rightarrow \frac{2\rho^2\sigma_X^2\sigma_Y^2}{n} \\
	\end{aligned}
	\end{equation} 
	We have 
	%	\begin{equation}\label{eq:poolsampVarasypm}
	%	\begin{aligned}
	%	&E[S_X^2] = \sigma_X^2, \var[S_X^2] \rightarrow \frac{2\sigma_X^4}{n}\\
	%	&\cov(S_X^2, S_Y^2) \rightarrow \frac{2\rho^2\sigma_X^2\sigma_Y^2}{n} \\
	%	&E[S_Y^2] = \sigma_Y^2, \var[S_Y^2] \rightarrow \frac{2\sigma_Y^4}{n}\\
	%	\end{aligned}
	%	\end{equation}
	\begin{equation}\label{eq:poolsampVarasypm}
	\begin{aligned}
	& \sqrt{n}\left[\left( \begin{array}{c}
	S_{X, 1}^2\\
	S_{Y, 1}^2\\
	\end{array}\right)
	-
	\left( \begin{array}{c}
	\sigma_X^2\\
	\sigma_Y^2\\
	\end{array}\right)
	\right]
	\stackrel{d.}{\longrightarrow} 
	N\left[
	\left(\begin{array}{c}
	0\\
	0\\
	\end{array} \right), 
	2\left(
	\begin{array}{cc}
	\sigma_X^4 &\rho^2\sigma_X^2\sigma_Y^2 \\
	\rho^2\sigma_X^2\sigma_Y^2  &\sigma_Y^4 \\
	\end{array}
	\right)
	\right] 
	\end{aligned}
	\end{equation}
	If we let $g(x) = x^{-\frac{1}{2}}$, and apply $\delta$-method to equation 
	(\ref{eq:poolsampVarasypm}), we obtain
	\begin{equation}\label{eq:invSasymp}
	\begin{aligned}
	& \sqrt{n}\left[\left( \begin{array}{c}
	S_X^{-1}\\
	S_Y^{-1}\\
	\end{array}\right)
	-
	\left( \begin{array}{c}
	\sigma_X^{-1}\\
	\sigma_Y^{-1}\\
	\end{array}\right)
	\right]
	\stackrel{d.}{\longrightarrow} 
	N\left[
	\left(\begin{array}{c}
	0\\
	0\\
	\end{array} \right), 
	\frac{1}{2}\left(
	\begin{array}{cc}
	\sigma_X^{-2} &\rho^2\sigma_X^{-1}\sigma_Y^{-1} \\
	\rho^2\sigma_X^{-1}\sigma_Y^{-1}  &\sigma_Y^{-2} \\
	\end{array}
	\right)
	\right] 
	\end{aligned}
	\end{equation}
	It follows from equation (\ref{eq:invSasymp}) that $\cor(S_X^{-1}, S_Y^{-1}) \rightarrow 
	\rho^2$.
	
	
	
	\subsection{Conclusion and discussion}
	
%	\textbf{State the major findings} \\
	This article discusses the relationship between population correlation $\rho$ and the 
	corresponding test statistics correlation $\rho_T$. We investigate $\rho_T$ for test statistics 
	of the form $(\frac{\bm a^T\bm X}{S_X}, \frac{\bm a^T\bm Y}{S_Y})$ (see equation 
	(\ref{eq:teststat})), where the denominator is the standard error of the numerator. Assuming 
	independence between $(\bm a^T\bm X, \bm a^T\bm Y)$ and $(S_X, S_Y)$, we derive the formula for 
	test statistics correlation $\rho_T$, and show that $\rho_T$ may not equal population 
	correlation $\rho$.  
	
	In two group comparison setting, we conclude that $\rho_T = \rho$ when $S_X$ (or $S_Y$) is 
	constant with respect to $\bm X$ (or $\bm Y$). That is, $\rho_T = \rho$ under linear 
	transformation of $\bm X$ and $\bm Y$, which is the case for two sample $z$-test. However, when 
	$S_X$ (or $S_Y$) is a function of $\bm X$ (or $\bm Y$), as is the case of two
	sample $t$-test, this equality may not hold. For two sample $t$-test, we prove that 
	$\rho_T=\rho$ only if the null in equation (\ref{eq:hypotheses}) is true for all the tests 
	considered, and that $|\rho_T|\leq |\rho|$ otherwise. In the case where one test is true null 
	and the other true alternative, $\rho_T$ is directly proportional to $\rho$, while when both 
	tests are true alternatives, $\rho_T$ is a quadratic function of $\rho$.
	
%	\textbf{State the practical meaningness of the findings}\\
	We note that cares need to be taken when estimating correlations between test statistics.
	In gene expression analysis, the two sample $t$-test \citep{barry2008statistical, 
		efron2007correlation,qiu2005correlation} or moderated 
	$t$-test \citep{wu2012camera} are used to calculate test statistics for DE detection, and the 
	sample correlation (after treatment effects 
	nullified) are used to adjust for correlation between those test statistics.
	Our study shows that, however, for DE genes, $\rho_T$ may be overestimated when
	two genes are positively correlated, and underestimated when they are negatively correlated. If 
	there are true DE genes whose expression 
	levels are correlated in either way, the VIF may not be accurately estimated in 
	\cite{wu2012camera}, resulting in biased test
	for their enrichment analysis as we will see in Chapter \ref{chap3}. Our results also indicate 
	that the variance of 
	$\rho_T$ may also be overestimated in
	\cite{efron2007correlation}, which leads to larger variation in estimating their conditional 
	FDP.
	
%	\textbf{	Acknowledge the study?s limitations \\}
	Theorem \ref{thm:teststatcor} and the subsequent results hold when the following two 
	assumptions are met: 1) the test statistic has the of 
	the form $\bm a^T\bm X/S_X$, and 2) $\bm a^T\bm X$ and $S_X$ are independent. In practice, both 
	assumptions are vulnerable.
	The test statistic may take different forms, depending on many factors such as the nature of 
	the data (RNA-Seq or microarray), the 
	experimental design structure, and the statistical hypothesis to be tested. The independence 
	assumption between $\bm a^T\bm X$ and $S_X$ are 
	unlikely to hold unless the statistic is derived from two sample $t$-test for normally 
	distributed random variables. Therefore, the 
	application of Theorem \ref{thm:teststatcor} is somewhat limited. Yet one goal of this study is 
	to raise awareness that the equality of $\rho_T$ 
	and $\rho$ should not be taken for granted. In the future, we will explore the relationship 
	between $\rho_T$ and $\rho$ for more general cases and for other types of statistics. 
	
	The R codes for reproducing results in this paper are available at Github: 
	\url{https://github.com/zhuob/CorrelatedTest}.
	% \textbf{Make suggestions for further research\\ }


	\subsection*{Acknowledgement}			
	We thank Sarah Emerson for valuable comments and suggestions in method development and 
	manuscript preparation. This article is part of doctor dissertation of BZ under the supervision 
	of YD.
	




	\newpage
	%	\section{Appendix}
