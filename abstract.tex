
Differential expression (DE) analysis is a key task in gene expression study, because it uncovers 
the association between expression levels of a gene and the covariates of interest.
This dissertation pertains to two particular aspects of DE analysis---identifying stably expressed 
genes for count normalization and accounting for correlation between DE test statistics in gene set 
test. Over the past few years, RNA-Sequencing (RNA-Seq) has become the tool of choice for measuring 
gene expression. Data generated from RNA-Seq experiments are the focus of this thesis. 

Identifying stably expressed genes is useful for count normalization and DE analysis. We examined 
RNA-Seq data on \howmanySamples biological samples from \howmanylab different experiments conducted 
by different labs, and identified genes that are stably expressed across samples, treatment 
conditions, and experiments. We fit a Poisson log-linear mixed-effect model to the count data, and 
decomposed the total variance into between-sample, between-treatment and between-experiment 
variance components. The variance 
component analysis that we explore here is a first step towards understanding the sources and 
nature of the RNA-Seq count variation. The stability ranking of genes, when quantified a numerical 
stability measure, is dependent on several factors: the reference biological samples used, the 
technology used to measure gene expression, and the specific stability measure. Since DE is 
measured by relative frequencies, we argue that DE is a relative concept. We advocate using an 
explicit reference gene set for count normalization to improve interpretability of DE results, and 
recommend using a common reference gene set when analyzing multiple RNA-Seq experiments to avoid 
potential inconsistent conclusions.


We investigate the relationship between correlation among test statistics and the underlying 
correlation of observed data. For testing multiple genes or gene sets, pooling DE test 
statistics together is a frequently used idea and the correlation among test statistics needs to be 
taken into account. The sample correlation of observed data are often used to approximate the 
test statistics correlation. We show, however, that such an approximation is only valid under 
limited settings. In particular, we derive a formula for correlation between test statistics when 
they take a specific form, and as a special case, we present the exact expression of test 
statistics correlation for equal-variance two sample $t$-test statistic under bivariate 
normal assumption. We conclude that test statistics correlation is no more greater in absolute 
value than the underlying correlation of observed data (normally distributed) in the context of 
equal-variance two sample $t$-test.

Competitive gene set test is a widely used tool for interpreting high-throughput biological 
data,such as gene expression and proteomics data. It aims at testing categories of genes for 
enriched association signals in a list of genes inferred from genome-wide data. Most 
conventional enrichment testing methods ignore or do not properly account for the widespread 
correlations among genes, which, as we show, can result in inflated type I error rates and 
power loss. We propose a new framework, \OurMethod, for gene set test based on a mixed effects 
quasi-likelihood model, where the data are not required to be Gaussian. Our method effectively 
adjusts for completely unknown,	unstructured correlations among the genes. It uses a score test 
approach and allows for analytical assessment of $p$-values. Compared to existing methods such 
as GSEA and CAMERA, our method enjoys robust and substantially improved control over type 1 
error and maintains good power in a variety of correlation structure and association settings. 
We also present two real data analysis to illustrate our approach.