\pagebreak
\newpage

\section{Conclusions}\label{ch:conclusion}

\subsection{Summary of the Dissertation}
Detecting differentially expressed genes plays a central role in gene expression analysis. The 
statistical questions we seek to address in this thesis help both ``pre" DE analysis---finding 
stably expressed genes for RNA-Seq count normalization, and ``post" DE analysis---investigating 
correlation among test statistics (for DE) and understanding the effect of treatments on sets of 
genes (biological pathways).

%\textbf{Major findings in chapter 2}\\
In Chapter \ref{chap1} we advocate quantifying RNA-Seq gene expression stability by a numerical 
measure applied to a large number of data sets. We show that traditional house-keeping genes are 
not necessarily stably expressed when examined by a numerical stability measure. 
%To find stably expressed genes for Arabidopsis, we 
%collected gene expression data from 24 RNA-Seq experiments conducted by different labs. We divided 
%the experiments into three groups based on the tissue type used, and identified three sets of 
%stably expressed genes accordingly. 
We emphasize that the ranking of genes in terms of numerical stability depends on multiple factors: 
the reference sample set used, the technology used for measuring gene expression, and the specific 
numerical stability measure used. Because DE is measured by relative frequencies, we argue that DE 
is a relative concept, and that making the reference gene set explicit can improve the 
interpretability of DE results.  When comparing differential expression among multiple experiments, 
we suggest using a common set of genes as reference to avoid inconsistent conclusions. For count 
normalization of Arabidopsis RNA-Seq data sets,  we recommend  using the top 1000 stably expressed 
genes chosen from the appropriate reference sample (multi-tissue, leaf or seedling, etc.).

In Chapter \ref{chap2} we explored the relationship between population correlation $\rho$ and test 
statistics correlation $\rho_T$ (see Section \ref{subsec:generalsetup} for details about $\rho$ and 
$\rho_T$) under bivariate setting. The statistics investigated are of the form $T_X = \bm a^T\bm 
X/S_X$. Under the assumption of independence between denominator and numerator of $T_X$, we show 
that $\rho_T$ does not necessarily equal $\rho$. For normally distributed 
random variables, we give exact formula of $\rho_T$ when $T$ is the test statistic for 
equal-variance two sample $t$-test. The formula (see equation (\ref{eq:limitT})) suggests that 
$\rho_T$ is a function of both $\rho$ and signal-to-noise ratio, and we conclude that $|\rho_T|\leq 
|\rho|$ for two sample $t$-test of normal data. In the context of gene expression analysis, the two 
sample $t$-test is a common choice for 
detecting DE genes, and sample correlation has been used to approximate test statistics 
correlation \cite{barry2008statistical,efron2007correlation,wu2012camera}. Our theoretical 
derivation shows, however, that correlation between test statistics will be over-estimated (in 
absolute value) by sample correlation. This will have an impact on statistical inferences such as 
false discovery rate control or gene set tests that are intended to adjust for test statistics 
correlation by sample correlation.

The \OurMethod, a new gene set test framework we proposed in Chapter \ref{chap3}, is motivated by 
the observation (in Chapter \ref{chap2}) that correlation between test statistics may not equal 
correlation between expression levels. It~uses difference in mean as gene level statistics 
whose correlation are exactly the same as sample correlation, and works for testing gene set in 
two group comparison experiment. It has the flexibility to allow for DE genes in the 
background set, and does not require the data to be normally distributed. It effectively adjusts 
for completely unknown and unstructured correlations among genes. 
 We demonstrate by simulation study that \OurMethod~holds type I error size 
correctly, and maintains good power for all correlation structures we examined. We apply 
\OurMethod~to two real data sets and find meaningful gene sets that are otherwise overlooked by 
methods such as GSEA or CAMERA.


\subsection{Future Work}
To identify stably expressed genes, our model (equation (\ref{eq:GLMM})) still needs to estimate an 
initial set of normalization factors, which requires that we have to make assumptions about the 
relative fold changes between samples. This kind of circular dependence seems unavoidable. We use 
an one-step iteration to reduce the dependency of the results on the initial normalization factors. 
In the future, we can examine the numerically stably expressed genes by evolutionary genetics 
methods, and ask whether there is connection between measures of expression stability and measures 
of genotypic/phenotypic features such as directional selection \citep{sabeti2006positive}.
On the other hand, it would also be interesting to model the random effect terms more accurately. 
In Chapter \ref{chap1} the random effect terms are assumed to be normally distributed, which is 
appropriate 
if the purpose is just to estimate their corresponding variance components. A more careful 
examination of individual data sets suggests that the between sample variances differ across 
experiments. Therefore, it may be beneficial to build a prior distribution of the random effect 
terms when analyzing a new data set. 

The investigation in Chapter \ref{chap2} is just a first attempt towards understanding correlation 
between  test statistics. Our results suggest that test statistics correlation may not be 
approximated 
by sample correlation. In the future, we are interested in finding analytical expressions of 
correlation for test statistics of different hypothesis testing procedures, and as a result, 
improving the accuracy of FDP estimation based on the work of \citet{efron2007correlation}. 

In Chapter \ref{chap3}, we showed that \OurMethod~adequately adjusts for correlation among test 
statistics. However, everything comes at a price. 
It is important to choose appropriate DE measure so that correlations between test statistics can 
be estimated from the samples. Our strategy is to use linear combination of the data (e.g., 
difference in mean under two group comparison, see Chapter \ref{chap3}) as gene level test 
statistics, which may not be suitable for all cases. In many gene expression experiments, 
researchers  might use more complicated design to study factors of interest.  When the gene level 
test  statistics are derived from such a design based statistical analysis (e.g., generalized 
linear  models), it is often difficult to estimate correlation among test statistics from data. In 
the future, we can either generalize \OurMethod~to accommodate complicated design structures, or 
develop techniques that can estimate correlation between test statistics under 
more general statistical models.  










