\section{Accounting for correlations in competitive gene set
	test for improved interpretation of genome-scale data}\label{chap3}

	\begin{abstract}
		Competitive gene set test is a widely used tool for interpreting high-throughput biological 
		data,
		such as gene expression and proteomics data. It aims at testing categories of genes for 
		enriched
		association signals in a list of genes inferred from genome-wide data. Most conventional 
		enrichment
		testing methods ignore or do not properly account for the widespread correlations among 
		genes,
		which, as we show, can result in inflated type I error rates and power loss. We propose a 
		new
		framework, \OurMethod, for gene set test based on a mixed effects quasi-likelihood model, 
		where the
		data are not required to be Gaussian. Our method effectively adjusts for completely unknown,
		unstructured correlations among the genes. It uses a score test approach and allows for 
		analytical
		assessment of $p$-values. Compared to existing methods such as GSEA and CAMERA, our method 
		enjoys
		robust and substantially improved control over type 1 error and maintains good power in a 
		variety of
		correlation structure and association settings. We also present two real data analysis to 
		illustrate
		our approach.
	\end{abstract}
	
	%
	%  Please place your key words in alphabetical order, separated
	%  by semicolons, with the first letter of the first word capitalized,
	%  and a period at the end of the list.
	%
	%	
	%	\begin{keywords}
	%		%Colostrum; Milk; Milk oligosaccharide; Non-human mammal.
	%		mixed effects; quasi-likelihood; gene set test; correlation
	%	\end{keywords}
	%	
	%	\maketitle
	
	\subsection{Introduction}\label{section:introduction}
	
	%	\textbf{What is enrichment analysis? Why would people care about that?}\\
	\textit{Gene set test} is a statistical framework of studying the association between a test
	set---a \textit{prior} set consisting of biologically related genes---and a set of genes that 
	are
	significantly correlated with treatment or experimental design variables. A key task of gene
	expression analysis involves the detection of differentially expressed genes. Differential
	expression (DE) analysis  evaluates each individual gene separately, and therefore it fails to
	provide insight into the relation between treatment variables and the prior gene set under 
	study.
	Gene set test helps researchers better understand the underlying biological processes in terms 
	of
	ensembles of genes.
	
	%	\textbf{What are the differences between self-contained and competitive test? And how does 
	%they	work?}\\
	Depending on the definition of the null hypothesis, there are two types of gene set test
	\citep{goeman2007analyzing}: the \textit{self-contained} test and the \textit{competitive} 
	test. A
	self-contained test examines a set of genes by a fixed standard without reference to other 
	genes in
	the genome 
	\citep{goeman2004global,goeman2005testing,tsai2009multivariate,wu2010roast,huang2013gene}. A 
	competitive test compares DE genes in the test set to those not in the test set 
	\citep{tian2005discovering,wu2012camera,yaari2013quantitative}. Many methods, regardless of the 
	type of test, perform a three-stage analysis
	\citep{khatri2012ten}: on the first stage, a \textit{gene-level statistic} is calculated for 
	each
	gene in the whole genome to measure the association between the expression profiles and the
	experimental design variables; such gene-level statistic includes, among others,
	\textit{signal-to-noise ratio} \citep{subramanian2005gene}, \textit{ordinary $t$-statistic}
	\citep{tian2005discovering} or \textit{moderated $t$-statistic} \citep{Smyth2004moderated},
	\textit{log fold change} \citep{kim2005page} and \textit{$Z$-score} 
	\citep{efron2007correlation}. On
	the second stage, a \textit{set-level statistic} is obtained by utilizing the gene-level 
	statistics
	from the first stage and their membership with respect to the test set (i.e., whether the gene
	belongs to the test set). Examples of the set-level statistic are \textit{enrichment score}
	\citep{subramanian2005gene}, \textit{maxmean statistic} \citep{efron2007testing}, and statistic
	derived from convoluted distribution of gene-level statistics \citep{yaari2013quantitative}, to 
	name
	a few. On the third stage, a $p$-value is assigned to the test set by comparing the set-level
	statistic to its reference distribution. The competitive gene set test is much more popular 
	among
	genomic literatures \citep{goeman2007analyzing, gatti2010heading}.  
	
	%Competitive gene set test \citep{goeman2007analyzing} is a gene expression analysis that 
	%compares
	%differential expression (DE) for genes in the test set to that for genes not in the test set. 
	%Most
	%competitive gene set tests, as described by \cite{barry2008statistical}, are typically 
	%two-stage
	%procedures:  on the second stage, a $p$-value is reported from the The test set may represent
	%biological pathways or network, or some other grouping based on biological knowledge. 
	%Incorporating
	%such prior information of the grouping makes it easier for biologists to interpret the results 
	%of
	%DE analysis.
	
	%	\textbf{Independent gene set test} \\
	Many competitive gene set tests rely on independence of gene-level statistics which further 
	requires  
	independence among gene expression levels. Those tests are parametric or rank-based procedures
	that assume the gene-level statistics to be independent and identically distributed, or gene
	permutation procedures that generate the same approximate null for the set-level statistics. For
	example, PAGE  \citep{kim2005page} conducts one-sample $z$-test by comparing the mean of 
	gene-level
	statistics (i.e., the mean of log fold changes) in the test set to a normal distribution under 
	the
	null. The $2\times 2$ contingency-table-based tests examine the significance of the test set by
	dichotomizing the outcomes of DE analysis and cross-classifying the genes according to whether 
	they
	are indicated as DE and whether they are in the test set (see \cite{huang2009bioinformatics} 
	for a
	review and references therein). sigPathway \citep{tian2005discovering} and ``\gen" in the limma
	package \citep{Smyth2004moderated} evaluate the set-level $p$-values by permuting gene labels.
	However, tests assuming independence of genes may result in inflated false discovery rate
	\citep{efron2007testing,goeman2007analyzing, 
	gatti2010heading,wu2012camera,yaari2013quantitative},
	as genes within a gene set are often co-expressed and function together.
	
	%	\textbf{Tests that account for inter-gene correlation}\\
	A handful of methods have been proposed to account for inter-gene correlation in competitive 
	gene
	set test. One attempt is to evaluate the set-level statistic by permuting the biological sample
	labels \citep{subramanian2005gene, efron2007testing}. Permuting sample labels
	does not require an explicit understanding of the underlying correlation structure among genes 
	and
	thus protects the test against such correlation. Since permuting sample labels is 
	computationally
	inefficient, \citet{zhou2013empirical} proposed an analytic approximation to permutations for
	set-level score statistics, which preserves the essence of permutation gene set analysis with
	greatly reduced computational burden. However, an unavoidable problem arising from sample
	permutation approach is that it implicitly alters the null hypothesis being tested and it is
	therefore difficult to characterize the null and the alternative hypotheses
	\citep{goeman2007analyzing, khatri2012ten, wu2012camera}. Another attempt is to use set-level
	statistic that directly includes inter-gene correlation estimated from the data. For example, 
	CAMERA
	\citep{wu2012camera} calculates a \textit{variance inflation factor} (VIF) from sample 
	correlation
	(after the treatment effect removed), and then incorporates it into their set-level statistics 
	to
	account for inter-gene correlations. QuSAGE \citep{yaari2013quantitative}, which is a recent
	extension to CAMERA, also used the same VIF in their test procedure to adjust for 
	inter-gene correlations. The VIF is a crucial factor and valid estimation of it
	relies on the assumption that correlation between any two gene-level statistics are almost the 
	same as
	correlation between their corresponding expression levels. \citet{barry2008statistical} showed 
	by simulation
	that this assumption holds for several gene-level statistics
	(e.g., $t$-statistic, Wald-type statistic for regressing expression on censored time-to-event 
	data
	through a Cox proportional hazards model). However, this assumption is likely to be problematic 
	when a fraction of genes are 
	truly DE, in which case the correlation among gene-level
	statistics (e.g., $t$-statistics) can be badly estimated by sample correlation 
	(\thepapertobefinished). 
	% The CAMERA procedure thus may be too conservative for controling type I error in the presence 
	%of
	%DE genes, as will be demonstrated in our simulation study. 
	
	
	% (This is a self-contained gene set test) \cite{huang2013gene} uses a multivariate linear
	%regression model in which the inter-gene correlation are explicitly modeled by a working 
	%covariance
	%matrix. 
	
	%	\textbf{What do we propose?} \\
	We propose a new framework for enrichment analysis that we will call Mixed Effects 
	Quasi-Likelihood
	Enrichment Analysis (\OurMethod). Our idea is motivated by the discrepancy between correlations
	among expression levels and those among gene-level statistics caused by the presence of DE 
	genes.
	To tackle such discrepancy, we use differences in mean as gene-level statistics for a two group
	comparison experiment. We model the covariance of gene-level statistics by two variance 
	components,
	one attributable to correlations among samples after treatment effect removed, and the other
	attributable to the DE effect associate with the treatment. The benefit of quasi-likelihood is 
	that
	the data are not required to be Gaussian. Our method effectively adjusts for completely unknown,
	unstructured correlations among the genes. \OurMethod~uses a score test approach and allows for
	analytical assessment of $p$-values. Compared to existing methods including GSEA and CAMERA,
	\OurMethod~enjoys robust and improved control over type I error and maintains good power in a
	variety of correlation structure and association settings. 
	
	%	\textbf{What is the plan of this paper?} \\
	The rest of the paper is organized as follows: in Section \ref{section:methods} we describe the
	methodology and then the simulation setup of \OurMethod, and summarize related existing 
	methods; in
	Section \ref{section:results} we present results from comparison of \OurMethod~to other existing
	methods by simulation study, and illustrate the application of our method by two real data 
	sets; in
	Section \ref{section:conclusion} we conclude and also specifies the future work.
	
	
	
	\subsection{Methods}\label{section:methods}
	We consider a gene expression (e.g. RNA-Seq or microarray) experiment, in which we compare the 
	expression levels of samples from two groups: a treatment group with $n_1$ samples referred to 
	as
	``cases" and a control group with $n_2$ samples referred to as ``controls" ($n_1,n_2\ge 3$). 
	Suppose
	the expression levels of a set of $m$ genes are observed for each sample. An unknown subset of
	these genes are DE between cases and controls, with varying sign and magnitude of DE effects. 
	The
	genes are also allowed to have (negatively or positively) correlated expression levels. In
	enrichment analysis, we are interested in a pre-defined set of genes, for example, from a known
	pathway or given by a functional annotation term from a database such as KEGG
	\citep{kanehisa2000kegg} or GO \citep{ashburner2000gene}. Our goal is to test whether this known
	gene set is enriched with differential expression signals. Let $\bm G$ be an $m$-dimensional 
	vector
	defining the gene set of interest, where $G_i=1$ if and only if the $i^{th}$ gene is in the set 
	and
	$G_i=0$ otherwise. Our analysis will condition on $\bm G$ and test if $\bm G$ is associated with
	enhanced DE effects. In the following sections, we will first construct a hierarchical model 
	for the
	gene expression data incorporating possible correlations among the $m$ genes, from which we will
	derive a quasi-likelihood model for the gene-level DE statistics jointly for all the genes. 
	Based on
	this model, we will then present our enrichment test, and discuss its connections with CAMERA.
	Finally, we will describe our simulation studies used to evaluate our method. For the rest of
	\textbf{Methods}, our presentation of the method is conditional on $\bm G$ unless otherwise
	indicated.
	
	\subsubsection{\OurMethod}
	\paragraph{A hierarchical model for the gene expression data}\label{subsection:YModel}
	We will start by presenting the hierarchical model for the observed gene expression data, which
	will incorporate the following features. 
	Firstly, for a given sample, the expression levels of different genes are allowed to be
	correlated. We further assume that the correlation structure is the same across samples. 
	Secondly,
	different genes may have different baseline expression levels, where ``baseline" refers to the
	average among controls. Thirdly, for any given gene, its mean expression level in the treatment
	group can be either higher, lower or the same compared to the control group, depending on 
	whether
	the gene is up-regulated, down-regulated, or not DE. For the genes that are differentially
	expressed, their DE effects are modeled additively and are allowed to have heterogeneous signs 
	and
	magnitudes. Finally, given a gene, and its DE effect, the expression level is allowed to vary
	independently across samples, which captures measurement error and sample-level variability.
	
	To present our model formally, we first introduce some notation. Let $n=n_1+n_2$ be the total
	sample size. Let $\bm X$ be an $n$-dimensional known vector of 1's and 0's denoting the 
	case-control
	membership of the samples, with $X_i=1$ for a case and $X_i=0$ for a control. Let $\bm Y$ be an 
	$m$
	by $n$ matrix representing the expression data, in which each column is the expression profile 
	for a
	sample and $Y_{ij}$ ($1\le i\le m, 1\le j\le n$) is the expression level of sample $j$ at gene
	$i$. Let $\mu_i$ ($1\le i\le m$) be the baseline expression level for gene $i$. The quantities 
	$\mu_i$'s are
	treated as nuisance parameters and as we will see later do not contribute to our analysis. Let 
	$\bm
	\Delta=(\Delta_1, \cdots, \Delta_m)^T$ be a vector for the additive DE effects for the genes. 
	Gene
	$i$ is not DE if $\Delta_i=0$, up-regulated if $\Delta_i >0 $ and down-regulated if 
	$\Delta_i<0$. We
	model $\bm \Delta$ as a random effect, for which we will detail our assumptions later. Given 
	$\mu_i$
	and $\Delta_i$, the mean expression level for the control group and the treatment group are
	$\mu_i$ and $\mu_i+\Delta_i$, respectively. Given these means, the noise in the observed 
	expression
	data for the $j^{th}$ sample is denoted by the mean zero error vector
	$\epsilon_j=(\epsilon_{1j},\cdots,\epsilon_{mj})^T$, $1\le j\le n$. We assume
	$\bm\epsilon:=(\epsilon_1,\cdots,\epsilon_m)$ to be independent of $\bm \Delta$ and to have mean
	zero. Without loss of generality, we also assume $\mbox{Var}(\epsilon_{ij})=1$ for all genes and
	samples. For a real gene expression data set typically not satisfying this assumption, we can
	standardize the data by each gene to ensure that its empirical variance equals one before 
	implementing
	our method (see Appendix for more detail). For the covariance structure of $\bm \epsilon$, we 
	assume 
	\begin{gather}
	\text{$\epsilon_{j_1}$ and $\epsilon_{j_2}$ are independent, \;\;$j_1\ne
		j_2$},\label{eq:gene_indep}\\
	\mbox{Cov}(\epsilon_j|\bm G)=\bm C, \;\;1\le i\le n,\label{eq:gene_cor}
	\end{gather}
	where $\bm C$ is an $m$ by $m$ inter-gene correlation matrix shared by all samples and is
	generally unknown.
	
	Putting these elements together, we obtain the following model for the expression data $\bm Y$
	given $\bm X$ and $\bm G$ 
	\begin{align}
	\label{eq:Ymodel}
	Y_{ij} = \mu_i + X_j\cdot\Delta_i + \epsilon_{ij},
	\end{align}
	for $1\le i\le m, 1\le j\le n$. The term $\bm G$ enters this model via $\Delta_i$ and possibly 
	$\mu_i$.
	
	\paragraph{Assumptions on the DE effects $\Delta_i$}\label{subsection:DeltaModel}
	Conditional on $\bm G$, we assume that the $\Delta_i$'s are mutually independent and come from
	either of the two distributions, $\mathscr{D}_1$ and $\mathscr{D}_2$, depending on whether 
	$G_i=0$
	or 1. We denote the expected values of $\mathscr{D}_1$ and $\mathscr{D}_2$ by $\beta_0$ and
	$\beta_0+\beta_1$, respectively, and their variances by $\sigma_1^2$ and $\sigma_2^2$, 
	respectively.
	It follows that 
	\begin{equation}
	\label{eq:Delta}
	E(\bm \Delta|\bm G)=\beta_0 + \beta_1 \bm G,\;\; \mbox{var}(\bm \Delta|\bm G) = \sigma_1^2\bm
	I_1+\sigma_2^2\bm I_2,
	\end{equation} 
	where $\bm I_1$ and $\bm I_2$ are diagonal matrices of dimension $m$ with $0$'s and $1$'s on 
	their
	diagonals. The $1$'s in the diagonal of $\bm I_1$ correspond to the genes with $G_i=1$ and 
	those for
	$\bm I_2$ to the genes with $G_i=0$.
	
	Aside from the conditions in equation (\ref{eq:Delta}) on the first two moments, we do not 
	impose
	any specific distributional assumptions such as normality on $\bm \Delta$. For example, the
	distribution of a given $\Delta_i$ can put positive mass on zero, which allows for the highly 
	likely
	event that some of the genes are not DE. To further motivate our general framework for $\bm 
	\Delta$,
	we present a simple model included by equation (\ref{eq:Delta}) as a special case. Suppose the 
	$m$
	genes are independently sampled to be either DE or not. The probability for gene $i$ to be DE is
	$p_t$ if $G_i=1$ or $p_b$ if $G_i=0$. For DE genes, their DE effects are sampled independently 
	from
	a common distribution with mean $\mu_\delta$ and variance $\sigma^2_\delta$. Under these
	assumptions, 
	\begin{equation}
	\label{eq:DeltaBinom}
	E(\Delta_i|\bm G) = p_i\mu_{\delta},\;\; \text{Var}(\Delta_i|\bm G)= p_i\sigma_{\delta}^2 +
	p_i(1-p_i)\mu_{\delta}^2,
	\end{equation}
	where $p_i=p_{t}$ if $G_i=1$ and $p_i=p_{b}$ if $G_i=0$. It can be shown that this model is a
	special case of equation (\ref{eq:Delta}).
	
	\paragraph{Model for gene-level statistics}\label{subsection:UModel}
	For each gene $i$, we consider the gene-level statistic $U_i$ given by 
	\begin{equation}
	\label{eq:U}
	U_i = \dfrac{\sum_{j: X_j=1}Y_{ij}}{n_1} - \dfrac{\sum_{j: X_j=0}Y_{ij}}{n_2},
	\end{equation}
	which is sample mean difference in the expression levels between cases and controls. Given our
	assumption that $\epsilon_i$ has variance 1, $U_i$ provides a DE metric for gene $i$. We will
	construct a quasi-likelihood model for $\bm U=(U_1,\cdots,U_m)^T$ by deriving the mean and
	covariance structures of $\bm U$ from the model for $\bm Y$ described in Sections
	\ref{subsection:YModel} and \ref{subsection:DeltaModel}. We first observe that combining 
	equations
	(\ref{eq:U}) and (\ref{eq:Ymodel}) yields
	\begin{equation} 
	U_i = \Delta_i + \eta_i, \text{where } \eta_i = \dfrac{1}{n_1}\sum_{j: X_j=1}\epsilon_{ij}-
	\dfrac{1}{n_2}\sum_{j: X_j=0}\epsilon_{ij}.
	\end{equation}
	It can be shown based on equations (\ref{eq:gene_indep}), (\ref{eq:gene_cor}) and 
	(\ref{eq:Delta})
	that
	\begin{gather}
	E(\bm U|\bm G) = \beta_0+\beta_1 \bm G,\label{eq:U_mean}\\
	\Sigma:=\mbox{Var}(\bm U|\bm G) = \sigma_0^2\bm C + \sigma_1^2\bm I_1+\sigma_2^2\bm
	I_2,\label{eq:U_var}
	\end{gather}
	where $\sigma_0^2=1/n_1+1/n_2$ is a known parameter. We note that in equation (\ref{eq:U_var}), 
	the
	covariance structure of $\bm U$ has three components, a component with $\bm C$ which accounts 
	for
	the contribution from sample-level noise $\bm \epsilon$, and two additional components from the 
	DE
	effect $\bm \Delta$. It is noteworthy that both the $\bm C$ component and the $\bm \Delta$
	components contribute to the variance of $U_i$'s, whereas only the $\bm C$ component 
	contributes to
	the correlation among $U_i$'s.
	
	\paragraph{The set-level test statistic}\label{subsection:MEQL}
	For a competitive gene set test, it is often unclear what the hypothesized null is and what is
	being tested \citep{barry2008statistical,wu2012camera}. In our approach, to detect patterns of 
	the
	DE signals in the gene set of interest that stand out compared with genes not in the set, we 
	test
	$H_0: \mathscr{D}_0=\mathscr{D}_1$ against $H_1: \mathscr{D}_0\ne\mathscr{D}_1$. For example, 
	for
	the special scenario given by equation (\ref{eq:DeltaBinom}), this amounts to testing 
	$p_{b}=p_{t}$
	against $p_{b}\ne p_{t}$.
	To construct the test statistic, we focus on the part of the alternative space where
	$E(\mathscr{D}_0)\ne E(\mathscr{D}_1$), or equivalently $\beta_1\ne 0$. We first consider the 
	less
	interesting case with uncorrelated genes, in which $\bm C$ equals $\bm I$, an $m$-dimensional
	identity matrix. Under the quasi-likelihood model for $\bm U$ given in Section
	\ref{subsection:UModel},  the quasi-score statistic for $\beta_1$ has the form $S \propto \bm
	G^T(\bm U-\hat\beta_0\bm 1_m)$, where $\hat\beta_0=\overline{U}$ is an estimate for $\beta_0$ 
	and
	$\bm 1_m$ is a $m$-dimensional vector of 1's. To perform a quasi-score test, one would divide 
	$S^2$
	by its estimated variance under $H_0$ and the assumption that $\bm C=\bm I$. The resulting test
	statistic is 
	\begin{equation}
	T_{\text{u}} = \dfrac{S^2}{\widehat{\mbox{Var}}_{0, \bm C=\bm I}(S|\bm G)} = \dfrac{[\bm G^T(\bm
		U-\hat\beta_0\bm 1_m)]^2}{\bm G^T(\bm I-\bm H)\bm G}, 
	\end{equation}
	where $\bm H = \dfrac{1}{m}\bm 1_m\bm 1_m^T$. The subscript ``u" stands for ``uncorrelated 
	genes".
	For the case of interest when inter-gene correlation is present, $\bm C$ is a non-trivial
	correlation matrix. We will again form our test statistic based on $S$. However, for the
	denominator of the statistic, the null variance of $S$ will be evaluated under the 
	quasi-likelihood
	model with non-trivial $\bm C$. By equation (\ref{eq:U_var}), the variance of $S$ is given by
	$\mbox{Var}(S|\bm G) = \bm G^T(\bm I-\bm H)\Sigma(\bm I-\bm H)\bm G$. Note that $H_0:
	\mathscr{D}_0=\mathscr{D}_1$ implies $\sigma_1^2=\sigma_2^2$. Thus, under $H_0$,
	$\Sigma:=\mbox{Var}_{0}(\bm U|\bm G)=\sigma_0^2\bm C+\sigma_1^2\bm I$, where 
	$\sigma_0=1/n_1+1/n_2$
	is known and $\sigma_1^2$ is an unknown parameter. To estimate $\sigma_1^2$ under $H_0$, we 
	observe
	that $\mbox{Var}_{0}(U_i)=\sigma_0^2+\sigma_1^2$ and use
	$\hat\sigma_1^2=\sum_{i=1}^m(U_i-\overline{U})^2/(m-1)-\sigma_0^2$. Therefore, assuming $\bm C$ 
	is
	known, we can obtain the \OurMethod~test statistic given by
	\begin{equation}
	T = \dfrac{S^2}{\widehat{\mbox{Var}}_{0}(S|\bm G)} = \dfrac{[\bm G^T(\bm U-\hat\beta_0\bm
		1_m)]^2}{\bm G^T(\bm I-\bm H)\hat{\bm\Sigma}(\bm I-\bm H)\bm G}, 
	\end{equation}
	where $\hat{\bm\Sigma}=(1/n_1+1/n_2)\bm C+\hat\sigma_1^2\bm I$ is a null estimate of $\bm 
	\Sigma$.
	Under suitable regularity conditions, significance of the test could then be assessed by 
	comparing
	$T$ to a $\chi^2_1$ distribution.
	
	In practice, the inter-gene covariance matrix $\bm C$ is usually unknown. Therefore we 
	substitute $\bm C$
	with $\hat {\bm C}$, the empirical covariance matrix of the expression data after controlling 
	for
	possible DE effects by centering the expression levels of cases and controls separately around
	zero. Formally, $\hat {\bm C}$ is given by $\hat
	C_{ik}=\dfrac{1}{n}\sum_{j=1}^n(Y_{ij}-\alpha_{ij})(Y_{kj}-\alpha_{kj})$ where
	$\alpha_{ij}=\sum_{j':X_{j'}=X_{j}}Y_{ij'}/\sum_{j'=1}^n1\{X_{j'}=X_{j}\}$ is the average 
	expression
	level at gene $i$ for all samples from the same group (either treatment or control) as sample 
	$j$.
	In real data sets, the number of genes, $m$, is usually much greater than the sample size $n$, 
	in
	which case $\bm C$ is a high-dimensional parameter that cannot be efficiently estimated by $\hat
	{\bm C}$. Interestingly, however, we find that the the test statistic $T$ relies not on
	the accurate estimation of the entire $\bm C$, but only on three parameters involving $\bm C$, 
	which
	can be much more realistically estimated by a moderate sample size. To demonstrate this, we
	re-arrange the order of the rows and columns of $\bm C$ to allow the partition $\bm
	C=\left[\begin{array}{cc}
	\bm C_{11} & \bm C_{12} \\       \bm C_{12}^T & \bm C_{22} \\      \end{array}\right] $,
	where $\bm C_{11}$ is the correlation matrix for genes in the test set, $\bm C_{22}$ is that for
	gene in the background set (i.e., the complement of the test set), and $\bm C_{12}$ is the
	cross-correlation matrix between the two classes of genes. (To be continued....)
	
	
	
	%	\subsection{Connection to current methods}
	%	\OurMethod~...
	%	CAMERA incorporates a VIF in their set-level statistics to account for inflation due to the
	%presence of inter-gene correlation. \OurMethod is closely related to their approach in that
	%set-level statistic in (REF)    
	
	%	\subsection{Other competitive gene set tests}
	\subsubsection{Simulation Methods}
	
	\paragraph{Simulation Setup}\label{subsection:simulation}
	In this section, we will specify the parameter setup for type I error and power simulations. 
	%Since a
	%standardization procedure is required by \OurMethod~for preprocessing data, we will simulate 
	%the
	%standardized expression levels for method illustration purpose. 
	Let $\bm Y_{j}$ be a vector denoting the expression profile of sample $j$ and 
	$\text{Cov}(Y_{i_1, j}, Y_{i_2, j})=\rho_{i_1,i_2}$ for any two genes $i_1$ and $i_2$. 
	We assume that genes have the same correlation if they are from the same category (whether the 
	test set or the background set): $\text{Cov}(Y_{i_1}, Y_{i_2})= \rho_1$ if
	genes $i_1$ and $i_2$ are both from the test set (i.e., $G_{i_1} = G_{i_2}=1$), 
	$\text{Cov}(Y_{i_1}, Y_{i_2}) =\rho_2$ if they are both from the background set (i.e., $G_{i_1} 
	=
	G_{i_2}=0$), and  $\text{Cov}(Y_{i_1}, Y_{i_2})= \rho_3$ if $i_1$ is from the test set and 
	$i_2$ is
	from the background set (i.e., $G_{i_1} =1,  G_{i_2}=0$). We examine five different correlation
	structures, listed as follows:
	
	\begin{enumerate}
		\item[(\aaCase):] $\rho_1 = \rho_2 = \rho_3 = 0$; that is, the genes are independent of each
		other.
		\item[(\cCase):] $\rho_1 = \rho_2 = \rho_3 = 0.1$; that is, all genes are correlated, with 
		an
		exchangeable correlation structure. 
		\item[(\aCase):] $\rho_1 = 0.1$, $\rho_2 = \rho_3 = 0$; that is, only the genes in the test 
		set
		are correlated. This corresponds to ... , and we envision what methods do well...
		\item[(\eCase):] $\rho_1 = 0.1$, $\rho_2 = 0.05$, $\rho_3 = 0$; that is, 
		genes are correlated within the test set and within the background set, but any two genes, 
		one
		from the test set and the other from the background set, are independent.
		\item[(\fCase):] $\rho_1 = 0.1$, $\rho_2 = 0.05$, $\rho_3 = -0.05$; that is, all genes are
		correlated, but the correlation between two genes depend on whether they belong to the test 
		set or
		not.
		%	\item[(g):] genes are correlated in the same way as those from a real data.
	\end{enumerate}
	By no means are such correlation structures intended to model the actual correlation structures 
	among gene expression levels.
	
	The simulations run as follows: first, we consider an entire gene set containing $m=500$ genes,
	of which $m_1 = 100$ genes are in the test set, and the
	remaining $m_2=400$ genes in the background set; second, we sample genes as DE with probability 
	$p_t$ in the test set and with probability $p_b$ in the background set, and for sampled DE 
	genes, we simulate the DE effect $\Delta$ from  a normal distribution $N(2, 1)$ (except in 
	Table \ref{table:power} we use $N(1, 0.5)$ to report calibrated power) and for non-DE genes we 
	set $\Delta= 0$ ; third, we set the ``true" mean expression values $\bm \mu_1 = \bm 0_m$ and 
	$\bm \mu_2 = \bm \Delta$, respectively,
	for the control and treatment groups; fourth, we simulate $n_1$ samples from $\text{MVN}(\bm 
	\mu_1,
	\bm \Sigma)$ for the control group and $n_2$ samples from $\text{MVN}(\bm \mu_2, \bm \Sigma)$ 
	for
	the treatment group, where the covariance $\bm \Sigma = \left[\text{Cov}(Y_{i_1},
	Y_{i_2})\right]_{m\times m}$ may be one of the correlation structures in (\aaCase)-(\fCase).
	
	Further assumptions on $p_t$ and $p_b$ will complete our generating model used in the type I 
	error and power simulations. (REF methods part about DE and no DE ....)
	We have mentioned in the Introduction part that the test statistics correlations among genes are
	not equal to their sample correlations when at least one gene is truly DE (under two sample
	$t$-test???). Therefore, if there are true DE genes in the entire gene set, approaches assuming
	almost equality of correlations among gene-level statistics and those among expression values 
	may
	not perform well. (a heads-up on how the 2 groups are different: if non-GO-term genes have DE) 
	To illustrate this point, we perform two groups of simulations for each of 
	(\aaCase)-(\fCase) correlation structures. In both type I error and power simulations, we set 
	the DE
	probability to be $0\%(S_0)$ in group $A_1$ and $10\%(S_0)$ in group $A_2$ for genes in the
	background set. In the type I error simulation, we have $p_t = p_b$ under the null. In the power
	simulation, we considered four different scenarios for the alternative hypothesis of the 
	presence of enrichment: for genes in the test set, we set DE
	probability to be $5\% (S_1), 10\%(S_2), 15\%(S_3)$ and $20\%(S_4)$ in group $A_1$, and 
	$15\%(S_1),
	20\%(S_2), 25\%(S_3)$ and $30\%(S_4)$ in group $A_2$. Table \ref{table:simusetup} summarizes the
	simulation setup for the two groups.
	
	% In both groups of simulation, we fixed the DE size $\delta$ for each alternative and the six
	%different correlation structures. 
	
	% (summarized in Table \ref{table:simusetup} ): we simulated expression data with no DE genes in
	%group $A_1$, that is, $p_b = 0$; and in group $A_2$, we simulated data sets with the same DE
	%probabilities for all genes---specifically, DE probabilities are the same for genes in the 
	%test set
	%and for those in the background set with $p_t= p_b = 0.2$. 
	
	
	%\begin{table}[!ht]
	%\centering
	%\caption{DE proportions in type I error and power simulations. $S_0$ repersents scenario for
	%type I error simulation. $S_1$-$S_4$ represent the four alternatives regarding power 
	%simulation. The
	%$p_b$ and $p_t$ are, respectively, the DE probablility for genes in the background set and for 
	%those
	%in the test set.}
	%\begin{tabular}{rp{0.8cm}p{0.8cm}p{0.1cm} p{0.5cm}p{0.5cm}p{0.5cm}p{0.5cm}p{0.5cm}}
	%\hline\hline
	%&\multicolumn{2}{c}{type I error} & & \multicolumn{5}{c}{power}  \\ 
	%\cline{2-3}         \cline{5-9}
	%& $S_0$ &$S_0$ & &$S_0$ &$S_1$ &$S_2$ &$S_3$ &$S_4$ \\
	%Group 	& $p_t$ & $p_t$ &  & $p_b$ & $p_t$ & $p_t$ & $p_t$ &$p_t$\\ \hline
	%$A_1$ & 0\% & 0\%  &  & 0\%    & 5\% & 10\% & 15\% & 20\%\\ 
	%$A_2$ & 10\% & 10\% &  & 10\%  & 15\%& 20\%& 25\% & 30\%\\ 
	%\hline\hline
	%%		\multicolumn{6}{p{8cm}}{}	 \\	
	%%		\multicolumn{6}{p{8cm}}{$p_t$: DE probablility for genes  }	 \\	
	%\end{tabular}
	%\label{table:simusetup}
	%\end{table}
	
	\begin{table}[!ht]
		\centering
		\caption{DE probability configurations in type I error and power simulations. $S_0$ is for
			type I error simulation. $S_1$-$S_4$ represent the four scenarios considered in power 
			simulations. $p_b$ and $p_t$ are the DE probability for genes in the background set and 
			that in the test set, respectively.}
		
		\begin{tabular}{rp{3cm}p{1cm}p{1cm}p{1cm}p{1cm}p{1cm}}
			\hline
			\multirow{2}{*}{Group} & 	Background & \multicolumn{5}{c}{DE prob. in test set 
			($p_t$)}\\
			&DE prob. ($p_b$)&$S_0$ &$S_1$ &$S_2$ &$S_3$ &$S_4$\\
		\hline
			$A_1$ & 0\% & 0\%  & 5\% & 10\% & 15\% & 20\%\\ 
			$A_2$ & 10\% & 10\% & 15\%& 20\%& 25\% & 30\%\\  
		\hline
		\end{tabular}
		
		\label{table:simusetup}
	\end{table}
	
	
	\paragraph{Other methods considered}
	
	We will compare \OurMethod~to~\HowmanyTest~previously proposed gene set tests: GSEA
	\citep{subramanian2005gene}, two versions of the CAMERA procedure ---\CMT~and
	\CMR~\citep{wu2012camera}, \gent~\citep{tian2005discovering}, 
	\genr~\citep{michaud2008integrative},
	and QuSAGE \citep{yaari2013quantitative}. Except \gent~and \genr, all methods incorporate 
	features intended for inter-gene correlation correction. GSEA calculates an enrichment score 
	for the test set by examining the ranking (according to some metric, for example, the 
	signal-to-noise ratio) of its member genes, and determines the significance of the enrichment 
	score by randomly permuting sample labels. \CMT~uses moderated $t$-statistics 
	\citep{Smyth2004moderated} as gene-level statistics and estimate a VIF to account for 
	inter-gene correlations in the set-level statistic, and \CMR~is the rank version of the \CMT.
	\genr~is a rank-based method assuming inter-gene independence, which is recommended by 
	\citet{tarca2013comparison} over a class of independence-assuming methods. \gent~is a 
	parametric 
	version of \genr, and in this simulation we use the moderated $t$-statistics as the gene level 
	statistics. 
	%	 The \gent~is slightly different from its original version of \cite{tian2005discovering} in
	%	that it uses moderated $t$-statistics rather than the ordinary $t$-statistics as gene-level
	%	statistics. \genr is also known as the rank version of \gent~\citep{wu2012camera}. 
	QuSAGE generates
	from $t$-test a probability density function (PDF) for each gene, combines the individual PDFs 
	using
	convolution, and quantifies enrichment of the test set with the convoluted PDF. 
	
	
	The software implementation is described as follows. The GSEA is modified from the original 
	R-GSEA script (\url{http://software.broadinstitute.org/gsea/index.jsp}) to accommodate single 
	gene set test.
	CAMERA and \genr~are implemented in the limma package \citep{smyth2005limma} in the Bioconductor
	project \citep{gentleman2004bioconductor}, QuSAGE is available in the Bioconductor package of 
	the
	same name, and \gent~is implemented by ourselves. (Move to discussion)Because GSEA and 
	\OurMethod~do not support linear models, the implementations are restricted to two-group 
	comparisons.
	
	
	%The \HowmanyTest~tests differ in one or more aspects, although all tests except QuSAGE follow 
	%the
	%three-stage paradigm described in Section~\ref{section:introduction}. For GSEA, the gene-level
	%statistics are the rankings of genes according to a ranking metric (we use signal-to-noise 
	%ratio,
	%the default metric in R-GSEA throughout this paper), then based on the rankings an enrichment 
	%score
	%for the test set is calculated, and the significance of the enrichment score is determined by
	%randomly permuting the sample labels. Both \CMT~and \gent~use the moderated $t$-statistics
	%\citep{Smyth2004moderated} as gene-level statistics, and determine whether the means of the
	%gene-level statistics are significantly different for genes in the test set versus genes in the
	%background set. The difference is how they evaluate the set-level statistics: \CMT~uses a
	%$t$-statistic that allows the gene-level statistics in the test set to be correlated by first
	%estimating a VIF, and then incorporating it into the $t$-statistic to adjust for inter-gene
	%correlation (see materials and methods section of \cite{wu2012camera}); \gent~accesses the
	%significance of the test set by comparing the observed set-level statistics to its null
	%distribution generated by permuting gene labels. \CMR~and MRGSE conduct a Wilcoxon-Mann-Whitney
	%rank sum test, and they amount to, respectively,  \CMT~and \gent~in that they compare the 
	%rankings
	%instead of the gene-level statistics themselves for genes in the test set to those for genes 
	%in the
	%background set. QuSAGE generates from $t$-test a probability density function (PDF) for each 
	%gene,
	%combines the individual PDFs using convolution, and quantifies gene-set activity with a 
	%complete
	%PDF. The complete PDF can be used to compare a baseline value for self-contained gene set 
	%test, or
	%to compare differences in expression levels between test set and background set in competitive
	%gene set test.  
	
	In terms of type I error control and power, we expect some of the six tests to have different
	performances between group $A_1$ and $A_2$ simulations under one or more correlation 
	structures. 
	
	
	\subsection{Results}\label{section:results}
	
	According to the simulation setup in Section \ref{subsection:simulation}, the test set is not
	enriched if DE probabilities are the same for genes in the test set and for those in the 
	background
	set (i.e., $p_t =0\%$ for group $A_1$ and $p_t = 10\%$ for group $A_2$), in which case we 
	evaluate
	the type I error. As to power, we set DE probability according to each of the alternative 
	scenarios
	$S_1$-$S_4$ (see Table \ref{table:simusetup}) and calculate the proportion of data sets for 
	which a
	test would reject at a given level $\alpha$. The results are based on \HowmanySimu~simulated 
	data
	sets. 
	
	
	
	\subsubsection{Type I error simulations}\label{subsection:typeIerror}
	
	\textbf{Messages: 1. Our method is well calibrated for all scenarios. 2. All of these others 
	methods have poor calibration in at least some of the scenarios. 3. The way type I error is 
	deviates for the nominal level aligns with our expectations for GSEA (permutation-based), 
	CAMERA (rank-based or parametric T-test), and methods assuming independence (GRSGE and SigPath)}
	
	We report the type I error simulation results for group $A_1$ and $A_2$ simulations. Figure
	\ref{fig:typeIerror} shows the uniform quantile-quantile (QQ) plots of $p$-values for the
	seven~approaches (\OurMethod, \gent, \genr, \CMT, \CMR, GSEA and QuSAGE) under each of the five
	correlation structures (each row of plots, from top to
	bottom, corresponds accordingly to correlation structures (\aaCase)-(\fCase)). 
	
	In group $A_1$ simulations (the left column of Figure \ref{fig:typeIerror}),  GSEA and
	\OurMethod~hold the size of type I error rate correctly for all five correlation structures, 
	with
	simulated $p$-values uniformly distributed on $[0, 1]$. The two versions of CAMERA control type 
	I
	errors correctly for correlation structures (\aaCase) and (\aCase), yet they are too 
	conservative
	for the case of (\cCase) and anti-conservative for correlation structures (\eCase) and (\fCase).
	\gent~and \genr~procedures have well-calibrated type I error for correlation structures 
	(\aaCase)
	and (\cCase), but are anti-conservative the case of (\aCase), (\eCase) and (\fCase). QuSAGE has 
	good
	type I error control for only (\aCase), and is too conservative for (\aaCase), (\eCase) and
	(\fCase), and anti-conservative for (\cCase).
	
	In group $A_2$ simulations (the right column of Figure \ref{fig:typeIerror}),  
	\OurMethod~continues
	to hold the size of type I error rate, whereas GSEA is skewed towards small $p$-values, under 
	all
	five correlation structures. The two versions of CAMERA control type I error rate correctly for
	(\aaCase) where genes are simulated to be independent, but may be liberal in other situations
	%(\CMT~is conservative in (\cCase)-(\fCase), while \CMR~is conservative in (\cCase) and 
	%(\aCase),
	%but anti-conservative in (\fCase)).
	\gent~and \genr~have similar trends for $p$-values as they do, respectively, in group $A_1$
	simulations. QuSAGE is conservative in (\cCase) but anti-conservative in the remaining four
	correlation structures.
	
	\textbf{Explain why this happens}\\
	\OurMethod~shows consistent accuracy for type I error control across all simulations, but the
	accuracy of the other \HowmanyTest~methods may be affected by two factors: the inter-gene
	correlation structures, and DE probability of each gene. \OurMethod~controls the size of type I
	error well because it uses difference in mean as gene-level statistic, and the correlations 
	between
	such statistics are exactly the same as correlations between the samples 
	(\thepapertobefinished).
	GSEA evaluates the enrichment score
	of a test set by generating its null distribution from sample permutation. When there's no DE 
	genes
	such as in the case of group $A_1$ simulations, GSEA performs extremely well since permuting 
	sample
	labels won't change the underlying correlation structure. When DE genes exist, however, sample
	permutation will destroy the inter-gene correlation structure, which explains the complete 
	failure
	of GSEA in controlling type I error for the case of group $A_2$ simulations. For CAMERA and 
	QuSAGE,
	the VIF of the gene-level statistics (moderated $t$-test in \cite{wu2012camera}) may be
	over-estimated when a fraction of genes are DE (\thepapertobefinished), and therefore the 
	set-level
	test statistic is under-estimated. The performances of related methods---QuSAGE and two 
	versions of
	CAMERA---are subject to the underlying correlation structures. (Considering remove this 
	sentence...)
	Moreover, the performance of CAMERA is complicated by the fact that the set-level statistic 
	takes
	into account only the inter-gene correlation in the test set without addressing that in the
	background set.
	
	Different from the five methods mentioned above, \gent~and \genr~rely on independence between
	genes. It's not surprising that such gene permutation based methods control type I
	error correctly when genes are ``equally-correlated": in (\aaCase) genes are simulated to be
	independent, and in (\cCase) genes are simulated to have an exchangeable correlation structure.
	However, both \gent~and \genr~fail to hold type I error size for the remaining three correlation
	structures. These simulations show that even small inter-gene correlations will result in 
	inflated
	type I error rate when the test does not account for inter-gene correlations.  
	
	%We note that both methods perform better in group $A_2$ as compared to their counterpart in 
	%group
	%$A_1$ simulation under each of (\aCase), (\eCase) and (\fCase) correlation structures. In group
	%$A_2$ where there are DE genes both in the test set and in the background set, the correlation
	%between the gene-level statistics is smaller (in absolute value) than between sample 
	%correlation.
	%Since the genes are simulated to be slightly correlated ($\rho_1=0.1, \rho_2 = 0.05, \rho_3 =
	%-0.05$), the correlation between the gene-level statistics are almost negligible for 
	%\gen~procedure
	%to work. 
	
	
	\begin{figure*}[!th]
		\begin{center}
			\includegraphics[width=15cm,height=0.6cm]{Figures/parallel_legend.eps}
			\includegraphics[width=15cm,height=4.1cm]{Figures/parallel_a.eps}
			\includegraphics[width=15cm,height=4.1cm]{Figures/parallel_b.eps}
			\includegraphics[width=15cm,height=4.1cm]{Figures/parallel_c.eps}
			\includegraphics[width=15cm,height=4.1cm]{Figures/parallel_d.eps}
			\includegraphics[width=15cm,height=4.1cm]{Figures/parallel_e.eps}
			%	\includegraphics[width=8cm,height=4cm]{Figures/c_SizePoint1_10PCT.eps}
			%	\includegraphics[width=8cm,height=4cm]{Figures/e_SizePoint1_0PCT.eps}
			%	\includegraphics[width=8cm,height=4cm]{Figures/e_SizePoint1_10PCT.eps}
			%	\includegraphics[width=8cm,height=4cm]{Figures/f_SizePoint1_0PCT.eps}
			%	\includegraphics[width=8cm,height=4cm]{Figures/f_SizePoint1_10PCT.eps}
		\end{center} 
		\caption{Uniform quantile-quantile plots for $p$-values by different methods. Each plot 
		from top to
			bottom corresponds to correlation structures (\aaCase)-(\fCase), respectively. The left 
			column is
			for group $A_1$ simulation, and the right column for group $A_2$ simulation (see Table
			\ref{table:simusetup} for detail). Results are based on
			\HowmanySimu~simulations.}\label{fig:typeIerror}
	\end{figure*} 
	
	
	\subsubsection{Power simulation}\label{subsection:power}		 
	
	\textbf{	Messages: 1. Correlation structure among genes does impact power (Fig. 2). 2. If 
	the genes are independent (not likely...), our method has better or similar power than 
	independence-assuming methods. GSEA is more powerful when there is no background DE, but fails 
	miserably otherwise. \\
		What puzzles us: In Fig. 2, structure b (rho=0.1 for all genes) yields higher power than 
		structure a (uncorrelated genes).}
	
	
	We compare the power of \OurMethod~to those of the other \HowmanyTest~methods under correlation 
	structure (\aaCase)
	in which genes are simulated to be independent. 
	Since some of these tests are not well calibrated at the sample size
	considered (see results in Section \ref{subsection:typeIerror}), we report calibrated power. For
	calibrated power, the critical value $c(\alpha)$ is chosen so that when the null hypothesis is 
	true,
	exactly $100\cdot\alpha\%$ of the resulting $p$-values are less than $c(\alpha)$; that is,
	$c(\alpha)$ is  the $\alpha$ quantile of null distribution of $p$-values, where the null
	distribution is generated from simulation. Calibrated power allows a more fair comparison among
	tests, as tests that are too conservative under the null hypothesis will have greater power due 
	to
	the tendency to produce small $p$-values, yet this apparent power does not truly distinguish 
	between
	the null and the alternative.  
	
	Table \ref{table:power} summarizes the calibrated power for the two groups of simulations (i.e.,
	$A_1$ and $A_2$ in Table \ref{table:simusetup}). % We only report the results for correlation
	%structure (\aaCase) where genes are simulated to be independent. %(for power comparisons under 
	%%the
	%other four correlation structures, see online supplementary materials...). %In Table
	%\ref{table:power} we set the DE size $\delta$ to be 0.05 and simulated data in the way that 
	%genes
	%in the background set are not DE (i.e., $p_b=0$). 
	For $A_1$ simulations, GSEA has the highest, and rank based methods (\genr~and \CMR)~have the
	lowest, calibrated power across all four alternative scenarios. \CMT, \gent~and \OurMethod~have 
	no
	systematic difference in the calibrated power. % Furthermore, when the DE probability is $10\%$ 
	%or
	%higher (i.e., the case of  $S_2$-$S_4$), both \CMT~and \OurMethod~have comparable calibrated 
	%power
	%to that of GSEA and \gent. In group $A_2$, \gent~continues to achieve the highest calibrated 
	%power
	%while GSEA shows virtually no power. \CMT~and \OurMethod~still have indistinguishable 
	%calibrated
	%power and both are better than \genr and \CMR.
	In group $A_2$ simulations, GSEA shows virtually no power. \OurMethod, \CMT, and \gent~ have
	indistinguishable calibrated power and are among the best.
	%though it is slightly inferior to the best calibrated power from QuSAGE. 	
	
	\begin{table*}[!ht]
		\centering
		\caption{Recalibrated power (standard error) for different methods. The powers are 
		summarized
			under four alternatives $S_1$-$S_4$ in each of the group $A_1$ and $A_2$ simulations 
			(see Table
			\ref{table:simusetup} for detail). Results are based on \HowmanySimu~simulations.
		}\label{table:power}
		%	\begin{tabular}{cl M{2.0cm}M{2.0cm}M{2.0cm}M{2.0cm}M{2.0cm}}
		\begin{tabular}{cp{3cm}p{1.5cm}p{1.5cm}p{1.5cm}p{1.5cm}p{1.5cm}}
			\hline
			Group & Method &$c(\alpha)$	& $S_1$ & $S_2$ & $S_3$	&$S_4$  \\ 
			\hline
			\multirow{7}{*}{$A_1$} & \OurMethod & 0.045 & 0.340 & 0.741 & 0.944 & 0.991 \\ 
			&	\genr  & 0.051 & 0.111 & 0.284 & 0.533 & 0.766 \\ 
			&	\gent & 0.049 & 0.344 & 0.744 & 0.947 & 0.992 \\ 
			&	\CMT  & 0.051 & 0.336 & 0.737 & 0.943 & 0.990 \\ 
			&	\CMR  & 0.053 & 0.108 & 0.280 & 0.519 & 0.758 \\ 
			&	GSEA & 0.051 & 0.517 & 0.894 & 0.989 & 0.999 \\ 
			&	QuSAGE & 0.028 & 0.385 & 0.784 & 0.959 & 0.995 \\ 
		\hline
			\multirow{7}{*}{$A_2$} &	MEQLEA & 0.050 & 0.180 & 0.478 & 0.777 & 0.939 \\ 
			&		\genr & 0.048 & 0.104 & 0.269 & 0.530 & 0.781 \\ 
			&		\gent & 0.049 & 0.175 & 0.473 & 0.773 & 0.936 \\  
			&		\CMT & 0.052 & 0.173 & 0.466 & 0.766 & 0.933 \\ 
			&		\CMR & 0.050 & 0.102 & 0.262 & 0.521 & 0.771 \\ 
			&		GSEA & 0.000 & 0.000 & 0.000 & 0.000 & 0.000 \\ 
			&		QuSAGE  & 0.000 & 0.021 & 0.127 & 0.387 & 0.692 \\ 
		\hline
		\end{tabular}
	\end{table*}
	
	
	
	Figure \ref{fig:power} shows for \OurMethod, the variations in power according to different
	correlation structures across four alternative scenarios $S_1$-$S_4$. For each correlation 
	structure
	and each alternative, we report the power (without recalibration) at a significance level of 
	0.05.
	The top is the power for group $A_1$, and the bottom for group $A_2$.  The powers under 
	correlation structures (\aaCase)
	and (\cCase) are very similar, and are among the highest under each of the four alternatives. 
	It's
	not surprising because they correspond to the simplest correlation structures: gene expression
	values in (\aaCase) are simulated to be independent and in (\cCase) are simulated to have the 
	same
	correlation 0.1. As the correlation structure becomes more complex, from (\aCase) to (\eCase) 
	then
	to (\fCase), the power decreases under every alternative scenario. The power under correlation
	structure (\fCase) is the lowest for both $A_1$ and $A_2$ simulations.% We also note that the 
	%power
	%decreases when DE genes are present in the background set as compared to the case where there 
	%are
	%no DE genes in the background set. (We might need the same DE size to illustrate this point.)
	
	\begin{figure}[!ht]
		\begin{center}
			\includegraphics[width=15cm,height=8cm]{Figures/powerA1pct.eps}
			\includegraphics[width=15cm,height=8cm]{Figures/powerA2pct.eps}
		\end{center} 
		\caption{Power for \OurMethod~under correlation structures (\aaCase)-(\fCase) of Section
			\ref{subsection:simulation}. The top corresponds to group $A_1$ simulations, and the 
			bottom to group
			$A_2$ simulations (see Table \ref{table:simusetup}). The error bars are the $95\%$ CIs 
			based on
			\HowmanySimu~simulations. }\label{fig:power}
	\end{figure} 
	
	\subsubsection{Real Data}\label{section:realdata}
	We applied \OurMethod~to two example data sets, and compared the lists of enriched gene sets to 
	those
	obtained by other three methods (GSEA, \CMT~and \genr). %In both examples, \OurMethod~were able 
	%to
	%identify more gene sets as enriched. 
	Our results lend credence to previous studies in finding potential gene sets correlated with
	Huntington's disease and those correlated with chromosome Y and Y bands in lymphoblastoid 
	cells.  
	
	\paragraph{Huntington's Disease Data}
	\textbf{Summary: 1. Our method yields more signals than CAMERA and GSEA, but fewer than MRGSE. 
	2. P-values by our method are generally less conservative than those by CAMERA. For genes with 
	small p-values, our method tends to be conservative than MRGSE, and less so than GSEA. 3. The 
	top 30 gene sets identified by our method contain some plausible biologically functions linked 
	to Huntington's disease.}\\
	
	We examined the Huntington's Disease (HD) RNA-Sequencing (RNA-Seq) data 
	\citep{labadorf2015rna}  to
	identify enriched gene sets that are potentially responsible for HD. The mRNA expression 
	profiles in human
	prefrontal cortex were obtained from 20 Huntington's Disease samples and 49 neurologically 
	normal
	controls.  Expression values were normalized and filtered as described in the methods section of
	\citet{labadorf2015rna}.	The data, containing $28,087$ genes, is available as a series 
	GSE64810 in the GEO database (\url{http://www.ncbi.nlm.nih.gov/geo/}). 
	For each gene, we adjusted for two covariates--age at death (DeathAge) and RNA Integrity Number 
	(RIN), as also done by \citet{labadorf2015rna}. We followed their strategy of treating the two
	covariates as categorical. Briefly, DeathAge was binned into intervals 0-45, 46-60, 61-75, 
	76-90 and
	90+,  and RIN was dichotomized as  $>$ or $\leq$ 7. We regressed the normalized expression 
	levels on 
	AgeDeath and RIN and use the resulting residuals as the \textit{covariate-adjusted expression 
	levels}.
	
	We performed enrichment analysis on the covariate-adjusted data using the MsigDB
	\citep{subramanian2005gene} C2 Canonical Pathways (February 5, 2016, data last accessed).
	The C2 Canonical Pathways have a collection of 1330 gene sets, with an average set size of
	50 (the set sizes range from 3 to 1028, and the median is 29). Since the genes in C2 are named 
	by
	HGNC symbols and by ensembl IDs in the HD expression data set, we converted the ensembl IDs in 
	the
	expression data into HGNC symbols using \textit{BioMart}
	(\url{http://uswest.ensembl.org/biomart/martview/}). We retained $26,941$ genes that have
	corresponding HGNC symbols. 
	
	
	
	We applied four test procedures (\OurMethod, GSEA, \CMT~and \genr) to run enrichment analysis 
	for
	the entire C2 Canonical Pathways, and compared the four tests in terms of resulting
	enriched gene sets. %\genr~has been reported to be the best method among four methods that
	%\cite{tarca2013comparison} considered in terms of FDR .
	We used the \FDR~\citep{benjamini1995controlling} procedure (\FDRabb) to control the false
	discovery rate (FDR) for multiple hypothesis testing (unless specified otherwise, all 
	$p$-values in
	Real Data Section \ref{section:realdata}were adjusted by \FDRabb~procedure). The 
	\FDRabb~procedure is used
	when the test statistics under the null have non-negative correlations 
	\citep{benjamini2001control}.
	We note that since many pathways have overlapped genes, the \FDRabb~procedure should be 
	appropriate
	in our study.
	
	In Figure \ref{fig:HDdatap} we plot $\log 10~p$-values of \OurMethod~against those of GSEA, 
	\CMT~and \genr.
	%Compared to GSEA, smaller $p$-values (e.g., less than 0.1) resulting from \OurMethod~are more 
	%likely
	%to cluster---in contrast to larger $p$-values; that is, \OurMethod~produces more small 
	%$p$-values
	%than GSEA does while \OurMethod~and GSEA do not differ much in producing larger $p$-values. 
	The $p$-values of \CMT~are overwhelmingly larger than their counterparts of GSEA or \OurMethod, 
	yet 
	smaller than those of \genr, even 
	if $p$-values between \OurMethod and other three methods are highly correlated (Pearson's 
	correlation 
	of $\log 10 ~p$ between \OurMethod~and GSEA, \CMT~and \genr~are 0.90, 0.96, and 0.87 
	respectively). 
	(Can we say this? we don't know the truth) This is consistent with our earlier simulation (see 
	results in simulation 
	section\ref{subsection:typeIerror}) that
	\CMT~could be too conservative. The $p$-values of \genr~are in general larger than the 
	corresponding
	$p$-values of \OurMethod. 
	\begin{figure}[!ht]
		\begin{center}
			\includegraphics[width=14cm,height=7cm]{Figures/MEQLEA_GSEA.eps}
			\includegraphics[width=14cm,height=7cm]{Figures/MEQLEA_Camera.eps}
			\includegraphics[width=14cm,height=7cm]{Figures/MEQLEA_MRGSE.eps}
		\end{center} 
		\caption{Pairwise comparisons of $p$-values for \OurMethod, GSEA, and \CMT. The $p$-values 
		are
			reported from enrichment test of each gene set in the C2 Canonical Pathway gene sets.
		}\label{fig:HDdatap}
	\end{figure} 
	
	%	\begin{figure}
	%		\begin{center}
	%			\includegraphics[width=5.5cm,height=5cm]{Figures/log10P_Camera.eps}
	%			\includegraphics[width=5.5cm,height=5cm]{Figures/log10P_GSEA.eps}
	%			\includegraphics[width=5.5cm,height=5cm]{Figures/log10P_GSEA_CAMERA.eps}
	%		\end{center} 
	%				\caption{raw p value plots for different methods, -log10(p  + 
	%1e-6)}\label{fig:HDdatalog10p}
	%	\end{figure} 
	
	Using \OurMethod, we found 89 significant signals out of the entire 1330 gene sets at FDR level
	of 0.05. GSEA found 3 enriched gene sets---2 of them were also among those 89 gene sets (the one
	that was not significant according to \OurMethod~had a $p$-value of 0.013 and FDR 0.100).
	\genr~found 387 gene sets which include all the 89 sets \OurMethod~identified, and \CMT~found 
	none.
	Originally, \citet{labadorf2015rna} used the same HD data set to conduct enrichment analysis 
	using
	topGo \citep{alexa2010topgo}. They found that the enriched gene sets they identified show a 
	clear
	immune response and inflammation-related pattern, including ``REACTOME INNATE IMMUNE SYSTEM, 
	PID IL4
	2PATHWAY", and ``PID NFKAPPAB CANONICAL PATHWAY". These three gene sets rank (by nominal
	$p$-values) 18,10 and 3 respectively in the 89 enriched gene sets.
	
	In Table \ref{table:top30}, we report the top 30 enriched gene sets (ordered by nominal $p$ 
	values)
	identified using \OurMethod. We also labeled the enriched gene sets from GSEA by ``$\ast$" in 
	the
	table. Many of our enriched gene sets have been shown to be closely related to HD pathogenesis. 
	For
	example, the top enriched gene set by \OurMethod,``PID SMAD2 3NUCLEAR PATHWAY", is responsible 
	for
	regulation of nuclear SMAD2/3 signaling. \citet{katsuno2010disrupted} showed that nuclear 
	SMAD2/3 are
	related to polyglutamine disease, which includes HD. The third enriched gene set, ``PID NFKAPPAB
	CANONICAL PATHWAY", is a canonical NF-kappaB pathway, and its dysregulation causes HD immune
	dysfunction \citep{trager2014htt}. Also, \citet{marcora2010huntington} found that reduced 
	transport
	of NF-kappaB out of dendritic spines and its activity in neuronal nuclei may contribute to the
	etiology of HD. 
	%HD mutation can reduce the transport of NF-kappaB out of dendritic spines and its activity in
	%neuronal nuclei and this reduction may contribute to the etiology of HD. 
	Another gene set, ``REACTOME INNATE IMMUNE SYSTEM", contributes to HD pathogenesis
	\citep{trager2014htt, labadorf2015rna}. %\cite{diamanti2013whole} showed that ``REACTOME
	% TRANSCRIPTIONAL ACTIVITY OF SMAD2 SMAD3 SMAD4 HETEROTRIMER", a gene set involved in 
	%transcriptional
	%activity of SMAD2/SMAD3:SMAD4 heterotrimer, is also enriched in their microarray study of HD
	%pathology from blood samples of R6/2 at manifest stage and wild type littermate mice. 
	%For AKT signaling pathway, ``BIOCARTA AKT PATHWAY", \cite{humbert2002igf} demonstrated that
	%huntingtin is a substrate of AKT and that phosphorylation of huntingtin by AKT is crucial to 
	%mediate
	%the neuroprotective effects of IGF-1. They also showed that AKT is altered in Huntington?s 
	%disease patients.  
	\citet{chiang2010modulation} demonstrated that the systematic downregulation of PPAR$\gamma$,
	related to ``BIOCARTA PPARA PATHWAY", seems to play a critical role in the dysregulation of 
	energy
	homeostasis observed in HD, and that PPAR$\gamma$ is a potential therapeutic target for this
	disease. %For ``REACTOME SIGNALING BY TGF BETA RECEPTOR COMPLEX",  
	%\cite{kandasamy2011transforming}
	%demonstrated that TGF-beta1 signaling appears to be a crucial modulator of neurogenesis in HD
	%pathology and it can be a promising target for endogenous cell-based regenerative therapy. 
	For ``PID P53 DOWNSTREAM PATHWAY", \citet{ghose2011regulation} showed the likely involvement of 
	NFkB
	(RelA), p53 and miRNAs in the regulation of cell death in HD pathogenesis. 
	

	\begin{table*}[!ht]
		%	\label{table:gender} 
		\centering 
		\caption{Enriched gene sets (ordered by nomial $p$-values) identified by \OurMethod~for HD 
		data.
			The $\hat\rho_1$,  $\hat\rho_2$ and $\hat\rho_3$, respectively, are the average 
			estimated sample
			correlation between genes in the test set, between genes in the background set, and 
			between two
			genes--one from the test set and the other from the background set. The enriched gene 
			sets are noted
			by ``$\ast$" for GSEA. No gene set was identified as enriched by \CMT~and all the 30 
			gene sets are
			also identified as enriched by \genr. For all methods, a gene set is called significant 
			when its FDR
			using \FDR~(\FDRabb) correction is $<0.05$. }
		%	\begin{tabular}{p{3in}M{0.5cm}M{0.7cm}M{0.7cm}M{0.7cm}M{1.5cm}M{1.5cm}M{0.8cm}} 
		\scalebox{0.83}{
		\begin{tabular}{p{3in}rrrrrrr}
			\hline
			Gene Set & Size & $\hat\rho_1$ & $\hat\rho_2$ & $\hat\rho_3$ & $p$-value & FDR & \\ 
			\hline
			PID SMAD2 3NUCLEAR PATHWAY & 79 & 0.063 & 0.013 & 0.015 & 5.8E-06 & 5.7E-03 & $\ast$ \\ 
			REACTOME YAP1 AND WWTR1 TAZ STIMULATED GENE EXPRESSION & 23 & 0.121 & 0.013 & 0.014 & 
			8.5E-06 & 5.7E-03 &  \\ 
			PID NFKAPPAB CANONICAL PATHWAY & 22 & 0.127 & 0.013 & 0.019 & 2.3E-05 & 1.0E-02 &  \\ 
			BIOCARTA NTHI PATHWAY & 23 & 0.130 & 0.013 & 0.023 & 6.2E-05 & 2.1E-02 &  \\ 
			BIOCARTA TID PATHWAY & 18 & 0.101 & 0.013 & 0.012 & 1.2E-04 & 2.2E-02 &  \\ 
			PID HIV NEF PATHWAY & 35 & 0.065 & 0.013 & 0.013 & 1.2E-04 & 2.2E-02 &  \\ 
			KEGG PATHWAYS IN CANCER & 311 & 0.028 & 0.013 & 0.010 & 1.3E-04 & 2.2E-02 &  \\ 
			PID MYC REPRESS PATHWAY & 60 & 0.057 & 0.013 & 0.013 & 1.9E-04 & 2.2E-02 &  \\ 
			BIOCARTA TOLL PATHWAY & 36 & 0.083 & 0.013 & 0.018 & 2.0E-04 & 2.2E-02 &  \\ 
			PID IL4 2PATHWAY & 59 & 0.081 & 0.013 & 0.010 & 2.0E-04 & 2.2E-02 &  \\ 
			KEGG TGF BETA SIGNALING PATHWAY & 82 & 0.055 & 0.013 & 0.011 & 2.2E-04 & 2.2E-02 &  \\ 
			BIOCARTA DEATH PATHWAY & 33 & 0.067 & 0.013 & 0.013 & 2.4E-04 & 2.2E-02 &  \\ 
			KEGG NOD LIKE RECEPTOR SIGNALING PATHWAY & 55 & 0.045 & 0.013 & 0.008 & 2.6E-04 & 
			2.2E-02 &  \\ 
			BIOCARTA CTCF PATHWAY & 23 & 0.083 & 0.013 & 0.015 & 2.8E-04 & 2.2E-02 &  \\ 
			ST TUMOR NECROSIS FACTOR PATHWAY & 28 & 0.031 & 0.013 & 0.014 & 3.2E-04 & 2.2E-02 &  \\ 
			BIOCARTA TNFR2 PATHWAY & 17 & 0.151 & 0.013 & 0.022 & 3.3E-04 & 2.2E-02 &  \\ 
			KEGG APOPTOSIS & 82 & 0.036 & 0.013 & 0.008 & 3.3E-04 & 2.2E-02 &  \\ 
			REACTOME INNATE IMMUNE SYSTEM & 209 & 0.039 & 0.013 & 0.009 & 3.3E-04 & 2.2E-02 &  \\ 
			PID HES HEY PATHWAY & 47 & 0.071 & 0.013 & 0.019 & 3.4E-04 & 2.2E-02 &  \\ 
			REACTOME DOWNSTREAM TCR SIGNALING & 31 & 0.082 & 0.013 & 0.011 & 3.7E-04 & 2.2E-02 &  
			\\ 
			PID TCPTP PATHWAY & 42 & 0.076 & 0.013 & 0.010 & 3.7E-04 & 2.2E-02 &  \\ 
			BIOCARTA 41BB PATHWAY & 14 & 0.110 & 0.013 & 0.023 & 3.9E-04 & 2.2E-02 &  \\ 
			PID FRA PATHWAY & 34 & 0.154 & 0.013 & 0.008 & 4.1E-04 & 2.2E-02 &  \\ 
			PID P53 DOWNSTREAM PATHWAY & 131 & 0.045 & 0.013 & 0.012 & 4.2E-04 & 2.2E-02 &  \\ 
			PID EPO PATHWAY & 34 & 0.069 & 0.013 & 0.013 & 4.3E-04 & 2.2E-02 &  \\ 
			BIOCARTA PPARA PATHWAY & 53 & 0.031 & 0.013 & 0.008 & 4.4E-04 & 2.2E-02 &  \\ 
			BIOCARTA EPONFKB PATHWAY & 11 & 0.068 & 0.013 & 0.010 & 4.7E-04 & 2.2E-02 &  \\ 
			BIOCARTA HIVNEF PATHWAY & 58 & 0.063 & 0.013 & 0.019 & 4.8E-04 & 2.2E-02 &  \\ 
			BIOCARTA CD40 PATHWAY & 13 & 0.165 & 0.013 & 0.026 & 4.8E-04 & 2.2E-02 &  \\ 
			BIOCARTA IL7 PATHWAY & 17 & 0.100 & 0.013 & 0.016 & 5.2E-04 & 2.3E-02 &  \\ 
			\hline
			%	\hline\hline
		\end{tabular}
	}
		\label{table:top30}
	\end{table*}

	
	
	%	
	
	
	\paragraph{Male vs Female Lymphoblastoid Cells Data}
	We analyzed the mRNA expression profiles from lymphoblastoid cell lines derived from 17 females 
	and
	15 males. \citet{subramanian2005gene} examined this data set with their GSEA method, testing the
	enrichment of the  cytogenetic gene sets (C1). C1 includes 24 sets, one for each of the 24 human
	chromosomes, and 295 sets corresponding to cytogenetic bands. For the comparison ``\text{male} 
	VS
	\text{female}", they expected to find gene sets on chromosome Y, not on chromosome X. We run
	enrichment analysis with the four tests (\OurMethod, GSEA,\CMT and \genr). In Table
	\ref{table:gender}, we summarized all the gene sets that are called significant at FDR level 
	$0.05$.
	Unanimously, three gene sets---``chrY", ``chrYq11" and ``chrYp11"---were found to be enriched 
	by all
	of the four methods. Interestingly, only \OurMethod~was able to identify another Y band, 
	``chrYp22",
	as enriched. In fact, these four gene sets are the only four pathways containing at least 3 
	genes in
	C1 and corresponding to chromosome Y or Y bands. \OurMethod~did not produce small $p$-value ($<
	0.01$) for the remaining three gene sets in Table \ref{table:gender}, which was just as 
	expected in
	that study.
	
	\begin{table*}[!ht]
		%	\label{table:gender} 
		\centering
		\caption{Enriched gene sets and their nominal $p$ values for lymphoblastoid cells data. 
		Reported
			are gene sets with FDR$<0.05$ for at least one of the \OurMethod, GSEA, \CMT~and 
			\genr~methods using
			\FDR(\FDRabb) procedure.}
		\begin{tabular}{p{2cm}p{1cm}p{2cm}p{2cm}p{3cm}p{2cm}p{0.5cm}} \hline
			Gene set & Size & \OurMethod & GSEA & \CMT & \genr \\ 		\hline
			chrY & 40 & 0.0E+00 & 0.0E+00 & 1.0E-05 & 5.9E-07 \\ 
			chrYq11 & 16 & 0.0E+00 & 0.0E+00 & 7.2E-08 & 8.5E-06 \\ 
			chrYp11 & 18 & 2.1E-15 & 0.0E+00 & 2.8E-04 & 5.1E-04 \\ 
			chrYp22 & 8 & 3.6E-04 & 1.2E-02 & 1.0E-02 & 1.3E-02 \\ 
			chr6 & 614 & 5.6E-02 & 6.0E-01 & 6.1E-01 & 2.1E-04 \\ 
			chr1 & 1104 & 6.1E-02 & 5.5E-01 & 6.3E-01 & 5.3E-05 \\ 
			chr12 & 571 & 8.7E-02 & 2.6E-01 & 4.0E-01 & 5.1E-09 \\ 
			\hline
		\end{tabular}
		\label{table:gender}
	\end{table*}
	
	\subsection{Conclusion and Discussion}\label{section:conclusion}
	
	(Conclusion) \OurMethod~is a mixed effects quasi-likelihood model for competitive gene set 
	test. It
	effectively adjusts for completely unknown, unstructured correlations among the genes. It uses a
	score test approach and allows for analytical assessment of $p$-values. Compared to existing
	approaches, \OurMethod~controls type I error correctly and maintains good power under different
	correlation structures.  
	
	
	(What we proposed) Under competitive gene set test framework, a number of methods have been
	proposed to account for correlation among genes. One approach is to evaluate the set-level 
	statistic
	by permuting sample labels to generate the null distribution, as adopted by the widely used GSEA
	\citep{subramanian2005gene}. However, sample permutation method has been criticized for  
	altering
	the null hypotheses being tested \citep{goeman2007analyzing, khatri2012ten}. Instead, CAMERA
	\citep{wu2012camera} proposed to correct for the correlation among genes by estimating a VIF
	directly from the data. This approach has also been used by \citet{yaari2013quantitative} in 
	their
	QuSAGE procedure. The major shortcoming with this approach is that it tries to estimate 
	correlations
	among gene-level test statistics directly from sample correlation (is it clear??). Zhuo and Di 
	have
	argued (unpublished work) that the correlations among gene-level statistics are not necessarily
	equal to those among samples due to the presence of DE genes. The estimated VIF could be biased
	without taking into account such a discrepancy and thus undermines the performance of CAMERA and
	QuSAGE. \OurMethod~avoids the discrepancy by using the differences in mean as gene-level 
	statistics
	for a two group comparison experiment. It models the covariance of gene-level statistics by two
	variance components, one attributable to correlations among samples after treatment effect 
	removed,
	and the other attributable to the DE effect associate with the treatment. We note that for
	\OurMethod, the estimation of covariance among gene-level statistics need not be exact:
	\OurMethod~uses a score test that involves linear combinations of the entries in the covariance
	matrix. The denominator in the score test statistic (REF EQ) can usually be accurately 
	approximated
	given the high dimensionality of the covariance matrix. \OurMethod~is based on quasi-likelihood,
	therefore it does not require normal assumption of expression data, and could be applied to both
	microarray and RNA-Seq experiments. 
	
	(Summarize the results) We compared \OurMethod~to other existing approaches in both simulation
	study and real data analysis. In the simulation study, we examined the performance of 
	\OurMethod~and
	other six method (\gent, \genr, \CMT, \CMR, GSEA and QuSAGE) in terms of type I error control 
	and
	power. We demonstrated that under a variety of correlation structures, \OurMethod~holds correct 
	type
	I error size and also maintains good power. In the real data analysis, \OurMethod~was able to
	identify more gene sets as enriched, some of which, in the corresponding studies, are 
	insightful yet
	not found by methods such as GSEA or CAMERA.  
	
	(Future work) Currently, \OurMethod~only supports enrichment test for two-group comparisons. In
	many gene expression experiments, however, researchers might use more complex design to study
	different comparisons of interest, in which case a linear model would be more appropriate. Our
	future work will focus on generalizing \OurMethod~to allow for more complicated design 
	structures. 
	
	The R codes for reproducing results in this paper are available at
	\url{https://github.com/zhuob/EnrichmentAnalysis}.
	%Using the terminology of \cite{khatri2012ten}, these methods generally fall into three 
	%categories:
	%\textit{over-representation analysis}, \textit{functional class scoring} and \textit{pathway
	%topology}. The over-representation analysis evaluates a fraction of genes among a set of
	%pre-selected interesting genes (e.g., differentially expressed genes between treatment versus
	%control samples). The test is usually conducted in the form of $2\times 2$ table, for example,
	%GOstat of \cite{klebanov2007multivariate} and GO:TermFinder of \cite{tian2005discovering}. 
	%However,
	%the over-representation analysis methods have inherent limitations such as information loss by
	%choosing arbitrary threshold (e.g., $p$-value $< 0.05$), or problematic assumption of 
	%independence
	%of genes \citep{goeman2007analyzing, wu2012camera}. The functional class scoring performs
	%three-stage analysis \citep{khatri2012ten}: on the first stage, a \textit{gene-level statistic}
	%that measures the association between the expression levels and the experimental design 
	%variables
	%is calculated for each gene; such gene-level statistics include, among others, signal-to-noise
	%ration \citep{subramanian2005gene}, moderated $t$ statistics \citep{Smyth2004moderated} and 
	%$Z$-score \citep{efron2007correlation}. On the second stage, a \textit{set-level statistic} is
	%calculated by using gene-level statistics and prior information about the test set (i.e., 
	%whether
	%the gene belongs to the set) as input. On the last stage, a $p$-value is assigned to the test 
	%set
	%by comparing the set-level statistic to its reference distribution.  (Rewrite this part)	The
	%pathway topology will not be discussed in this paper \citep{khatri2012ten,tarca2013comparison}.
	
	%\section{Conclusion}\label{section:conclusion}
	
	
	
	
	
	\subsection*{Acknowledgments}\label{section:acknowledgment}
	
	We thank Yanming Di, Sarah Emerson and Wanli Zhang for helpful discussion. 
	
	
	\subsection*{Conflict of interest statement.} None declared.
	
	\newpage
	
	\subsection*{Appendix}\label{section:appendix}
	
	
	\textbf{Standardization} 
	Standardization for each gene: first, we obtain the residuals by subtracting off the means 
	within
	each treatment group;
	\begin{equation}
	r_{ijk} = y_{ijk} - \sum_{j=1}^{n_k}{y}_{ijk}/n_k;
	\end{equation}
	then we calculate the pooled standard deviation from the residuals,
	\begin{equation}
	s_i = \textit{std}(r_{ijk});
	\end{equation}
	next we get the standardized expression by dividing the original expression levels by the
	standard deviation,
	\begin{equation}
	y^{\ast}_{ijk} = y_{ijk}/s_i
	\end{equation}
	We perform the standardization procedure to every gene in the data set.\\
	\textbf{Calculating covariance matrix for test statistics}
	First $E(\Delta_i) = E(Z_i\delta_i) = E(Z_i)E(\delta_i) = p_i\mu_{\delta}$. Next note that  
	\begin{equation}\notag
	\begin{aligned}
	\text{Var}(\Delta_i) & = E[(Z_i\delta_i)^2]- [E(Z_i\delta_i)]^2 \\
	& = \text{Var}(Z_i)[E(\delta_i)]^2 + \left[(EZ_i)^2 + 
	\text{Var}(Z_i)\right]\text{Var}(\delta_i) \\
	& =p_i\sigma_{\delta}^2 + p_i(1-p_i)\mu_{\delta}^2
	\end{aligned}
	\end{equation}
	
	Let $T_i=\bar{Y}_{i,2}-\bar{Y}_{i,1}$ be the difference in mean expression levels between the
	treatment group and the control group. We have 
	\[E(T_i) = E(\bar{Y}_{i,2})-E(\bar{Y}_{i,1}) = E(\Delta_i) = E(Z_i\delta_i) = p_i\mu_{\delta}\]
	The covariance between two genes $i_1$ and $i_2$ is given by (I HAVE CONCERNS HERE, IS IT VALID 
	TO
	ASSUME THAT DE EFFECTS ARE INDEPENDENT BETWEEN GENES?  WE SEE CO-EXPRESSION!! OR WE'VE ALREADY 
	TAKEN
	THAT INTO ACCOUNT BY CORRELATION BETWEEN GENES"), 
	
	\begin{equation}
	\begin{aligned}
	\text{Cov}(T_{i_1}, T_{i_2}) & = E\left[\text{Cov}(T_{i_1}, T_{i_2}|\Delta_{i_1}, \Delta_{i_2})
	\right] \\
	&  + \text{Cov}\left[E(T_{i_1}|\Delta_{i_1}), E(T_{i_2}|\Delta_{i_2})\right] \\
	& = E\left(\frac{1}{n_1}\rho_{i_1,i_2} + \frac{1}{n_2}\rho_{i_1,i_2}\right) +
	\text{Cov}(\Delta_{i_1}, \Delta_{i_2})\\
	& = \left(\frac{1}{n_1} + \frac{1}{n_2}\right)\rho_{i_1,i_2}
	\end{aligned}
	\end{equation}
	For gene $i$, the variance $\text{Var}(T_i) = \text{Var}(\bar{Y}_{i, 1}) + 
	\text{Var}(\bar{Y}_{i,
		2})$, with
	\[\text{Var}(\bar{Y}_{i, 1}) = \frac{1}{n_1}\] 
	\begin{equation}
	\begin{aligned}
	\text{Var}(\bar{Y}_{i, 2}) & = \frac{1}{n_2^2}\left[\sum_{j=1}^{n_2}\text{Var}(Y_{ij2}) +
	2\sum_{1\leq j_1<j_2 \leq n_2} \text{Cov}(Y_{ij_12}, Y_{ij_22})\right] \\
	& = \frac{1}{n_2}\text{Var}(Y_{ij2}) + \frac{n_2-1}{n_2} \text{Cov}(Y_{ij_12}, Y_{ij_22})\\
	& = \frac{1}{n_2}\left[E\left(\text{Var}(Y_{ij2}|\Delta_i)\right) +
	\text{Var}\left(E(Y_{ij2}|\Delta_i)\right)\right] \\ \text{~~~}
	&+\frac{n_2-1}{n_2}E\left(\text{Cov}(Y_{ij_12}, Y_{ij_22}|\Delta_i)\right) \\
	&+\frac{n_2-1}{n_2}\text{Cov}\left(E(Y_{ij_12}|\Delta_i), E(Y_{ij_22}|\Delta_i)\right) \\
	& = \frac{1}{n_2} + \text{Var}(\Delta_i)
	\end{aligned}
	\end{equation}
	Therefore $\text{Var}(T_i)  = \frac{1}{n_1} + \frac{1}{n_2} + \text{Var}(\Delta_i)$, and it 
	follows
	that
	\begin{equation}\label{eq:tvar}
	\text{Cov}(\bm T) =  \bm D + \sigma_2^2\bm C 
	\end{equation}
	where $\bm D$ is a diagonal matrix with $\text{Var}(\Delta_i) =p_i\sigma_{\delta}^2 +
	p_i(1-p_i)\mu_{\delta}^2$ as its $i$th diagonal element, and $\sigma_2^2 = \left(\frac{1}{n_1} +
	\frac{1}{n_2}\right)$.
	
	% latex table generated in R 3.2.3 by xtable 1.8-0 package
	% Fri Apr 22 13:30:45 2016
	\begin{table*}[ht]
		\centering
		\scalebox{0.8}{
		\begin{tabular}{lrrrrrrr}
			\hline
			case & \OurMethod & \genr & \gent & \CMT & \CMR & GSEA & QuSAGE \\ 
			\hline
			a0PCT & 0.056 & 0.049 & 0.051 & 0.049 & 0.047 & 0.049 & 0.078 \\ 
			a10PCT & 0.050 & 0.052 & 0.051 & 0.048 & 0.050 & 0.946 & 0.491 \\ 
			b0PCT & 0.059 & 0.050 & 0.051 & 0.000 & 0.000 & 0.048 & 0.000 \\ 
			b10PCT & 0.052 & 0.051 & 0.051 & 0.000 & 0.000 & 0.837 & 0.027 \\ 
			c0PCT & 0.056 & 0.513 & 0.517 & 0.051 & 0.044 & 0.051 & 0.052 \\ 
			c10PCT & 0.054 & 0.442 & 0.188 & 0.000 & 0.021 & 0.290 & 0.131 \\ 
			d0PCT & 0.059 & 0.586 & 0.594 & 0.114 & 0.104 & 0.051 & 0.106 \\ 
			d10PCT & 0.052 & 0.522 & 0.235 & 0.001 & 0.049 & 0.220 & 0.175 \\ 
			e0PCT & 0.058 & 0.674 & 0.679 & 0.213 & 0.197 & 0.053 & 0.203 \\ 
			e10PCT & 0.054 & 0.614 & 0.334 & 0.004 & 0.116 & 0.113 & 0.267 \\ 
			\hline
		\end{tabular}}
	\end{table*}
	
	
	\newpage