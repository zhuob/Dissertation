\newpage
\renewcommand{\thesubsection}{\Alph{subsection}}  % rename the chapters

\setcounter{table}{0}   % rename the tables
\renewcommand{\thetable}{\ref{app:supp}.\arabic{table}}

\begin{appendices}

	\subsection{Standardization}\label{app:standardization} 
	Standardization for each gene: first, we obtain the residuals by subtracting off the means 
	within
	each treatment group;
	\begin{equation}
	r_{ijk} = y_{ijk} - \sum_{j=1}^{n_k}{y}_{ijk}/n_k;
	\end{equation}
	then we calculate the pooled standard deviation from the residuals,
	\begin{equation}
	s_i = \textit{std}(r_{ijk});
	\end{equation}
	next we get the standardized expression by dividing the original expression levels by the
	standard deviation,
	\begin{equation}
	y^{\ast}_{ijk} = y_{ijk}/s_i
	\end{equation}
	We perform the standardization procedure to every gene in the data set.\\
	
	\subsection{Covariance matrix for test statistics}\label{app:covariance}
	Note that if we let $\delta_i$ be the DE size for gene $i$, then $E(\delta_i)= \mu_{\delta}$ 
	and $\var(\delta_i) = \sigma_{\delta}^2$. To prove the equation (\ref{eq:DeltaBinom}), we 
	introduce an additional random variable $Z_i$ for DE status, where 
	$Z_i\sim \text{Bernoulli}(1, p_t)$ if $G_i = 1$ and $Z_i\sim \text{Bernoulli}(1, p_b)$ if $G_i 
	= 0$. It follows that $\Delta_i =Z_i\delta_i$, and we have 
	\begin{equation}\notag
	E(\Delta_i|\bm G) = E(Z_i\delta_i|\bm G) = E(\delta_i)E(Z_i|\bm G)=  p_i\mu_{\delta},
	\end{equation}
	and 
	\begin{equation}\notag
	\begin{aligned}
	\text{Var}(\Delta_i|\bm G) & = E[(Z_i\delta_i)^2|\bm G]- [E(Z_i\delta_i|\bm G)]^2 \\
	& = \text{Var}(Z_i|\bm G)[E(\delta_i)]^2 + \left[(EZ_i|\bm G)^2 + 
	\text{Var}(Z_i|\bm G)\right]\text{Var}(\delta_i) \\
	& =p_i\sigma_{\delta}^2 + p_i(1-p_i)\mu_{\delta}^2..
	\end{aligned}
	\end{equation}
	If the gene-level test statistics $U_i$'s take the form of equation (\ref{eq:U}), then we have 
	$E(U_i|\bm G) = E(\Delta_i + \eta_i|\bm G)  = p_i\mu_{\delta}$. Next, note that the covariance 
	between two genes $i_1$ and $i_2$ is given by, 
	\begin{equation}\label{eq:app1}
	\begin{aligned}
	\text{Cov}(U_{i_1}, U_{i_2}|\bm G) & =\cov\left[(\Delta_{i_1} + \eta_{i_1}, \Delta_{i_2} + 
	\eta_{i_2})|\bm G\right] \\
	& = \text{Cov}(\Delta_{i_1}, \Delta_{i_2}|\bm G) + \text{Cov}(\eta_{i_1}, \eta_{i_2}|\bm G)\\
	& = \text{Cov}[(\dfrac{1}{n_1}\sum_{j: X_j=1}\epsilon_{i_1,j}-
	\dfrac{1}{n_2}\sum_{j: X_j=0}\epsilon_{i_1,j}, \dfrac{1}{n_1}\sum_{j: X_j=1}\epsilon_{i_2,j}-
	\dfrac{1}{n_2}\sum_{j: X_j=0}\epsilon_{i_2,j})|\bm G]\\
	& = \left(\frac{1}{n_1} + \frac{1}{n_2}\right)c_{i_1,i_2},
	\end{aligned}
	\end{equation}
	where $c_{i_1,i_2}$ is the corresponding entry in $\bm C$. Also 
	\begin{equation}\label{eq:app2}
	\begin{aligned}
		\var(U_{i_1}|\bm G) &= \var(\Delta_{i_1} + \eta_{i_1} |\bm G) \\
		& = \var(\Delta_{i_1}|\bm G) + \var(\dfrac{1}{n_1}\sum_{j: X_j=1}\epsilon_{i_1,j}-
		\dfrac{1}{n_2}\sum_{j: X_j=0}\epsilon_{i_1,j}|\bm G)\\
		& =\text{Var}(\Delta_{i_1}|\bm G) + \frac{1}{n_1} + \frac{1}{n_2}. 
	\end{aligned}
	\end{equation}
	Equation (\ref{eq:U_var}) immediately follows from equations (\ref{eq:app1}) and 
	(\ref{eq:app2}).	
	% latex table generated in R 3.2.3 by xtable 1.8-0 package
	% Fri Apr 22 13:30:45 2016
	
	
	
	\subsection{Supplimentary Tables}\label{app:supp}
	%	\subsection{Sources of the twenty-four RNA-Seq data sets}\label{app:24experiments} 
		\begin{table}[H]
			\centering
			\caption[Summary information for the 24 experiments used in Chapter 
			\ref{chap1}]{Summary of data sets used in the three groups: the seedling, the leaf and 
			the 
				multi-tissue groups.} 
			\label{app:24experiments}
			\scalebox{0.80}{
			\begin{tabular}{lrlrr} \hline
				GEO Number & Reference                              & Tissue                      
				&Sample Size     &Mapping Quality\tablefootnote{The number of mapped reads 
				divided 
				by 
					the total number of reads in the sample}($\geq$) \\ \hline
				GSE32202 & NA                                     & seedling                    & 
				6  & 
				84.30\% \\
				
				GSE37159 & NA                                     & seedling                    & 
				8  & 
				82.50\% \\
				%	GSE38400\tablefootnote{This data set is excluded in the analysis due to low 
				%mapping 
				%quality.} & \cite{zhu2013swi}                    & seedling                    & 
				%12 & 
				%45.90\%  \\
				GSE41766 & \cite{bai2012triple}                 & seedling                    & 6  
				& 
				85.20\% \\
				%	GSE43703\tablefootnote{This data set is excluded in the analysis due to low 
				%mapping 
				%quality.}  & \cite{liu2013translational}          & seedling                    & 
				%8 & 
				%22.50\% \\
				\tablefootnote{We chose all the 6 samples at the onset of treatment out of 
					the 
					42 samples.}GSE43865   & \cite{rugnone2013lnk}                & 
					seedling                    & 6 
				& 92.50\% \\
				GSE51119 & \cite{zhiponova2014helix}            & seedling                    & 10 
				& 
				70.50\% \\
				GSE51772 & \cite{oh2014cell}                    & seedling                    & 8  
				& 
				90.60\% \\
				GSE53078 & \cite{fan2014bhlh}                   & seedling                    & 4  
				& 
				86.10\% \\
				\tablefootnote{We chose the samples of ecotype Columbia from the 12 
					samples.}GSE60835 & 
				\cite{dong2014arabidopsis}           & seedling                    & 6  & 65.60\% 
				\\ 
				GSE66666 & \cite{capella2015arabidopsis}		&seedling 					&6  & 
				91.20\% \\ \hline
				GSE36626 & \cite{wollmann2012dynamic}           & leaves                      & 4  
				& 
				85.50\% \\
				\tablefootnote{We chose all samples at 6 hours post inoculation out of the 
					48 
					samples.}GSE39463 & \cite{maekawa2012conservation}       & 
					leaves                      
					& 12 & 
				74.50\% \\
				GSE48235 & \cite{liu2014different}              & leaves                      & 6  
				& 
				90.40\% \\
				\tablefootnote{Out of the 48 samples,  the total number of RNA-Seq samples 
					is 
					18.}GSE51304 & \cite{stroud2014non}                 & 
					leaves                      & 18 
					& 
				88.40\% \\
				\tablefootnote{Out of the 24 samples,  the total number of RNA-Seq samples 
					is 
					20.}GSE54677 & \cite{moissiard2014transcriptional}  & 
					leaves                      & 20 
					& 
				85.20\% \\ \hline
				GSE35288 & \cite{niederhuth2013transcriptional} & flower                      & 6  
				& 
				76.50\% \\
				GSE35408 & \cite{bai2012brassinosteroid}        & hypocotyl                   & 10 
				& 
				77.00\%    \\
				GSE52966 & \cite{chaiwanon2015spatiotemporal}   & primary root                & 18 
				& 
				87.20\% \\
				GSE56326 & NA                                     & carpels                     & 
				8  & 
				92.40\% \\
				GSE59167 & \cite{pallakies2014cle40}            & root tip tissue             & 11 
				& 
				87.10\% \\
				GSE59637 & \cite{mizzotti2014seedstick}         & inflorescences and siliques & 4  
				& 
				71.70\% \\
				GSE60183 & \cite{kimura2015flowering}           & epidermis					  & 6  
				& 
				62.30\% \\
				GSE61061 & \cite{macgregor2015seed}   	        & seed                        & 6  
				& 
				91.60\% \\
				GSE62799 & \cite{groth2014snf2}                 & aerial tissue               & 6  
				& 
				89.30\% \\
				GSE63355 & \cite{liu2015repair}                 & shoot apical meristem       & 16 
				& 
				87.90\%  \\ \hline
			\end{tabular}
		}
		\end{table}
		
		
	
	\begin{landscape}		
	\begin{table}[ht]
	%	 \renewcommand\tablename{\ref{app:supp}}  % give the table a new name format
		\caption[Type I error rates for different methods]{Type I error rates for different methods 
				(see Chapter \ref{chap3}). The type I error rates are summarized at a significant 
				level of 0.05, based on \HowmanySimu~simulated data sets.}
			\label{table:apptypeIerror}
		\centering
		\begin{tabular}{llrrrrrrr}
			\hline
		Group & Correlation & \OurMethod & \genr & \gent & \CMT & \CMR & GSEA & QuSAGE \\
			\hline
			\multirow{5}{*}{$A_1$}  & (a) & 0.056 & 0.049 & 0.051 & 0.049 & 0.047 & 0.049 & 0.078 
			\\ 
			 & (b) & 0.059 & 0.050 & 0.051 & 0.000 & 0.000 & 0.048 & 0.000 \\ 
			 & (c) & 0.056 & 0.513 & 0.517 & 0.051 & 0.044 & 0.051 & 0.052 \\ 
			 & (d) & 0.059 & 0.586 & 0.594 & 0.114 & 0.104 & 0.051 & 0.106 \\ 
			 & (e) & 0.058 & 0.674 & 0.679 & 0.213 & 0.197 & 0.053 & 0.203 \\ \hline
			\multirow{5}{*}{$A_2$} & (a) & 0.050 & 0.052 & 0.051 & 0.048 & 0.050 & 0.946 & 0.491 \\ 
			 & (b) & 0.052 & 0.051 & 0.051 & 0.000 & 0.000 & 0.837 & 0.027 \\ 
			 & (c) & 0.054 & 0.442 & 0.188 & 0.000 & 0.021 & 0.290 & 0.131 \\ 
			 & (d) & 0.052 & 0.522 & 0.235 & 0.001 & 0.049 & 0.220 & 0.175 \\ 
			 & (e) & 0.054 & 0.614 & 0.334 & 0.004 & 0.116 & 0.113 & 0.267 \\ 
			\hline
		\end{tabular}
	\end{table}
	
	\end{landscape}	

\end{appendices}
	



\newpage